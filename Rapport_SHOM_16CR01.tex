%%%%%%%%%%%%%%%%%%%%%%%%%%%%%%%%%%%%%%%%%%%%%%%%%%%%%%%%%%%%%%%%%%%
%%%%%%%%%%%%%%%%%%%%%%%%%%%%%%%%%%%%%%%%%%%%%%%%%%%%%%%%%%%%%%%%%%%
%           Rapport Contrat SHOM/LA   16CR01                      %
%%%%%%%%%%%%%%%%%%%%%%%%%%%%%%%%%%%%%%%%%%%%%%%%%%%%%%%%%%%%%%%%%%%
%%%%%%%%%%%%%%%%%%%%%%%%%%%%%%%%%%%%%%%%%%%%%%%%%%%%%%%%%%%%%%%%%%%
\documentclass[a4paper,11pt]{report}

\usepackage[utf8]{inputenc}
\usepackage[margin=2.5cm]{geometry}
\usepackage[hidelinks]{hyperref}
\usepackage{titlesec}
\usepackage{lipsum,etoolbox}% http://ctan.org/pkg/{lipsum,etoolbox}
\usepackage[utf8]{inputenc}

\usepackage[french]{babel}
\usepackage[T1]{fontenc}
\frenchbsetup{StandardLists=true} 
\usepackage{enumitem}

%%%%%%%%%%%%%%%%%%%%%%%%%%%%%%%%%%%%%%%%%%%%%%%%%%%%%%%%%%%%%%%%%%%
%%%%%package article 2D Margaux
%%%%%%%%%%%%%%%%%%%%%%%%%%%%%%%%%%%%%%%%%%%%%%%%%%%%%%%%%%%%%%%%%%%
\usepackage{graphicx}
\usepackage{multirow}
\usepackage{tabularx}
\usepackage{color}
\usepackage[fleqn]{amsmath}
\usepackage{amsfonts}
\usepackage{amssymb}
\usepackage{textcomp}
\usepackage{gensymb}
\usepackage{amsxtra}
\usepackage{wasysym}
\usepackage{isomath}
\usepackage{mathtools}
\usepackage{txfonts}
\usepackage{upgreek}
\usepackage{enumerate}
\usepackage{tensor}
\usepackage{pifont}
\usepackage{titlesec}
\usepackage[T1]{fontenc}
\usepackage{fancyhdr}

\usepackage{subcaption}
\usepackage[normalem]{ulem} %25/05
\usepackage{caption}%04/06
\usepackage{afterpage}%04/6
\usepackage{geometry}%05/06
\usepackage{wrapfig}%04/06
\usepackage{chngcntr}
%%%%FIN MH

%%%%%%%%%%%%%%%%%%%%%%%%%%%%%%%%%%%%%%%%%%%%%%%%%%%%%%%%%%%%%%%%%%%
%%%%%%%%  Chapter and Section Numbering (1)               %%%%%%%%%
%%%%%%%%  Francis                                         %%%%%%%%%
%%%%%%%%%%%%%%%%%%%%%%%%%%%%%%%%%%%%%%%%%%%%%%%%%%%%%%%%%%%%%%%%%%%
\usepackage{etoolbox}
\titleformat*{\section}{\large\bfseries}
\titleformat{\chapter}[hang]
    {\normalfont\Large\bfseries}{\chaptertitlename\ \thechapter:}{1em}{\Large}
    \titlespacing*{\chapter} {0pt}{50pt}{40pt}
\usepackage{lipsum}  
\usepackage{pdfpages}

%%%%%%%%%%%%%%%%%%%%%%%%%%%%%%%%%%%%%%%%%%%%%%%%%%%%%%%%%%%%%%%%%%%
%%%%%% Lib doc Lucie%%%%%%%%%%
%%%%%%%%%%%%%%%%%%%%%%%%%%%%%%%%%%%%%%%%%%%%%%%%%%%%%%%%%%%%%%%%%%%
%\usepackage[normalem]{ulem}
\usepackage{array}
%\usepackage{amssymb}
%\usepackage{graphicx}

%\usepackage[backend=biber,
%    style=numeric,
%    sorting=none,
%    isbn=false,
%    doi=false,
%    url=false,
%]{biblatex}\addbibresource{bibliography.bib} %Margaux 

%\usepackage{subfig}
%\usepackage{wrapfig}
%\usepackage{wasysym}
%\usepackage{enumitem}
\usepackage{adjustbox}
\usepackage{ragged2e}
\usepackage[svgnames,table]{xcolor}
%\usepackage{tikz}
\usepackage{longtable}
%\usepackage{changepage}
\usepackage{setspace}
\usepackage{hhline}
\usepackage{multicol}
\usepackage{tabto}
%\usepackage{float}
\usepackage{multirow}
%\usepackage{makecell}
%\usepackage{fancyhdr}
%\usepackage[toc,page]{appendix}
%\usepackage[hidelinks]{hyperref}
%\usetikzlibrary{shapes.symbols,shapes.geometric,shadows,arrows.meta}
%\tikzset{>={Latex[width=1.5mm,length=2mm]}}
%\usepackage{flowchart}\usepackage[paperheight=11.69in,paperwidth=8.27in,left=0.98in,right=0.98in,top=0.98in,bottom=0.98in,headheight=1in]{geometry}
%\usepackage[utf8]{inputenc}
%\usepackage[T1]{fontenc}
\TabPositions{0.5in,1.0in,1.5in,2.0in,2.5in,3.0in,3.5in,4.0in,4.5in,5.0in,5.5in,6.0in,}
\usepackage{chngcntr}
\counterwithin{figure}{section}
\counterwithin{table}{section}

%\urlstyle{same}
%%%%%% Lib doc Lucie%%%%%%%%%%

%%%%%%%%%%%%%%%%%%%%%%%%%%%%%%%%%%%%%%%%%%%%%%%%%%%%%%%%%%%%%%%%%%%
% Title Page
%%%%%%%%%%%%%%%%%%%%%%%%%%%%%%%%%%%%%%%%%%%%%%%%%%%%%%%%%%%%%%%%%%%
\title{
Modélisation des fines échelles internes\\ dans la région du détroit de Gibraltar.\\
\bigskip\bigskip
\large{Contrat SHOM / UPS 16CR01.}\\
\bigskip\bigskip\bigskip\bigskip\bigskip\bigskip
Groupe Croco, Laboratoire d'Aérologie}

\author{Lucie Bordois\footnote{Post-Doctorante.}, Margaux Hilt\footnote{Stagiaire M2 puis doctorante SDUEE 
Toulouse.},
Eric Chassignet\footnote{Professeur invité.},
Laurent Roblou, Cyril Nguyen, Francis Auclair\footnote{Permanents CNRS et UPS au Laboratoire d'Aérologie.}
}
%%%%%%%%%%%%%%%%%%%%%%%%%%%%%%%%%%%%%%%%%%%%%%%%%%%%%%%%%%%%%%%%%%%
% Bibliographie (Margaux)
%%%%%%%%%%%%%%%%%%%%%%%%%%%%%%%%%%%%%%%%%%%%%%%%%%%%%%%%%%%%%%%%%%%
\makeatletter %remove numbers in bibliography
\renewcommand\@biblabel[1]{}
\makeatother
%%%%%%%%%%%%%%%%%%%%%%%%%%%%%%%%%%%%%%%%%%%%%%%%%%%%%%%%%%%%%%%%%%%
%%%%%%%%%%%%%%%%%%%%%%%%%%%%%%%%%%%%%%%%%%%%%%%%%%%%%%%%%%%%%%%%%%%
%                   Début du document                             %
%%%%%%%%%%%%%%%%%%%%%%%%%%%%%%%%%%%%%%%%%%%%%%%%%%%%%%%%%%%%%%%%%%%
%%%%%%%%%%%%%%%%%%%%%%%%%%%%%%%%%%%%%%%%%%%%%%%%%%%%%%%%%%%%%%%%%%%
\begin{document} 

%%%%%%%%%%%%%%%%%%%%%%%%%%%%%%%%%%%%%%%%%%%%%%%%%%%%%%%%%%%%%%%%%%%
%%%%%%%%  Chapter and Section Numbering (2)               %%%%%%%%%
%%%%%%%%  Francis                                         %%%%%%%%%
%%%%%%%%%%%%%%%%%%%%%%%%%%%%%%%%%%%%%%%%%%%%%%%%%%%%%%%%%%%%%%%%%%%
\renewcommand{\thepage}{}
\renewcommand{\thechapter}{\Roman{chapter}}
\renewcommand{\thesection}{\arabic{section}}
\setcounter{secnumdepth}{4}
\newcommand*{\newpar}{%
  \ifsubsection\else\stepcounter{subsection}\fi
  \paragraph{}}
\setcounter{figure}{0}
\renewcommand{\thefigure}{\Roman{chapter}.\arabic{section}.\arabic{figure}}
\setcounter{table}{0}
\renewcommand{\thetable}{\Roman{chapter}.\arabic{section}.\arabic{table}}
%%% A conditional for knowing whether a \subsection
%%% command has been issued

\author{Lucie Bordois\footnote{Post-Doctorante.}, Margaux Hilt\footnote{Stagiaire M2 puis doctorante SDUEE Toulouse.},
Eric Chassignet\footnote{Professeur invité.},
Laurent Roblou, Cyril Nguyen, Francis Auclair\footnote{Permanents CNRS et UPS au Laboratoire d'Aérologie.}
}

%%%%%%%%%%%%%%%%%%%%%%%%%%%%%%%%%%%%%%%%%%%%%%%%%%%%%%%%%%%%%%%%%%%
% Bibliographie (Margaux)
%%%%%%%%%%%%%%%%%%%%%%%%%%%%%%%%%%%%%%%%%%%%%%%%%%%%%%%%%%%%%%%%%%%
\makeatletter %remove numbers in bibliography
\renewcommand\@biblabel[1]{}
\makeatother
%%%%%%%%%%%%%%%%%%%%%%%%%%%%%%%%%%%%%%%%%%%%%%%%%%%%%%%%%%%%%%%%%%%
%%%%%%%%%%%%%%%%%%%%%%%%%%%%%%%%%%%%%%%%%%%%%%%%%%%%%%%%%%%%%%%%%%%
%                   Début du document                             %
%%%%%%%%%%%%%%%%%%%%%%%%%%%%%%%%%%%%%%%%%%%%%%%%%%%%%%%%%%%%%%%%%%%
%%%%%%%%%%%%%%%%%%%%%%%%%%%%%%%%%%%%%%%%%%%%%%%%%%%%%%%%%%%%%%%%%%%

%%%%%%%%%%%%%%%%%%%%%%%%%%%%%%%%%%%%%%%%%%%%%%%%%%%%%%%%%%%%%%%%%%%
%%%%%%%%  Chapter and Section Numbering (2)               %%%%%%%%%
%%%%%%%%  Francis                                         %%%%%%%%%
%%%%%%%%%%%%%%%%%%%%%%%%%%%%%%%%%%%%%%%%%%%%%%%%%%%%%%%%%%%%%%%%%%%
\renewcommand{\thepage}{}
\renewcommand{\thechapter}{\Roman{chapter}}
\renewcommand{\thesection}{\arabic{section}}
\setcounter{secnumdepth}{4}
\newcommand*{\newpar}{%
  \ifsubsection\else\stepcounter{subsection}\fi
  \paragraph{}}
\setcounter{figure}{0}
\renewcommand{\thefigure}{\Roman{chapter}.\arabic{section}.\arabic{figure}}
\setcounter{table}{0}
\renewcommand{\thetable}{\Roman{chapter}.\arabic{section}.\arabic{table}}

%%% A conditional for knowing whether a \subsection
%%% command has been issued
\newif\ifsubsection

%%% Set the conditional to true after \subsection
%%% and reset the paragraph counter
\preto{\subsection}{\global\subsectiontrue\setcounter{paragraph}{0}}
%%% Reset it to false after \section
\preto{\section}{\global\subsectionfalse}

\counterwithin*{paragraph}{section}
\makeatletter
\renewcommand\paragraph{%
    \@startsection{paragraph}{4}{0mm}%
       {-\baselineskip}%
       {.5\baselineskip}%
       {\normalfont\normalsize\bfseries}}
\makeatother
\renewcommand{\theparagraph}{\alph{paragraph}.}

\maketitle
\newpage
	 \null
\newpage
 
{\setlength{\baselineskip}{0.8\baselineskip}
\tableofcontents\par}
\renewcommand{\thepage}{\arabic{page}}
\setcounter{page}{1}
%\tableofcontents

\maketitle
\newpage
	 \null
\newpage

\selectlanguage{french}
%%%%%%%%%%%%%%%%%%%%%%%%%%%%%%%%%%%%%%%%%%%%%%%%%%%%%%%%%%%%%%%%%%%
%%%%%%%%%%%%%%%%%%%%%%%%%%%%%%%%%%%%%%%%%%%%%%%%%%%%%%%%%%%%%%%%%%%
%%% Chapitre Introduction
%%%%%%%%%%%%%%%%%%%%%%%%%%%%%%%%%%%%%%%%%%%%%%%%%%%%%%%%%%%%%%%%%%%
%%%%%%%%%%%%%%%%%%%%%%%%%%%%%%%%%%%%%%%%%%%%%%%%%%%%%%%%%%%%%%%%%%%
\chapter{Introduction}
\label{chapitreintroduction}

\section{Contexte, généralités}
 \noindent\textbf{\textit{Extrait du Cahier des clauses techniques particulières du contrat de recherche Gibraltar 16CR01}}\\
Cette étude s'inscrit dans le cadre du projet stratégique PROTEVS (\textit{prévision océanique, turbidité, écoulement, vagues et sédimentologie}) qui vise l’amélioration des systèmes d’analyse et de prévision décrivant l’environnement marin en temps réel. Il s’agit de prendre en compte les couplages entre l’état de mer (vagues) et la circulation océanique (courants à plus grande échelle), et de réaliser des études et recherches pour la représentation des petites échelles, des paramètres biochimiques et de la dynamique sédimentaire. PROTEVS est un programme d’études amont (PEA 082401) financé par la DGA.\\

\noindent Dans ce contexte, les objectifs du contrat de recherche \textit{Gibraltar 16CR01} entre le SHOM et le laboratoire d'Aérologie  sont : 

\begin{itemize}[label=\textbullet]
 \item d’améliorer la connaissance des processus océaniques et côtiers impactant les opérations militaires (action continue de ``recherche amont``).
 \item d'étudier la faisabilité de systèmes permettant la reconstitution en temps réel (analyse) de l’environnement hydrodynamique (état de mer, courants, température, salinité) ainsi que de son évolution possible (prévision).
d’étudier des possibilités d’y intégrer des paramètres biogéochimiques (matière en suspension, mobilité des sols, transparence, visibilité).
\end{itemize}

\noindent  Le projet concerne l'amélioration d’un modèle numérique décrivant la circulation à petite échelle de l’océan. Il comporte trois tâches distinctes\footnote{La Tâche 3 a fait l'objet d'un avenant au contrat initial signé en 2016.}:
\begin{itemize}[label=\textbullet]
\item la première tâche consiste à implémenter un modèle numérique qui puisse totalement relaxer l’hypothèse d’hydrostaticité dans la région du détroit de Gibraltar et à procéder à une évaluation (coût calcul/performance) de la maquette numérique ainsi proposée.
\item la deuxième tâche concerne la généralisation du système de coordonnées verticales, à partir d’une approche ALE (\textit{Arbitrary Lagrangian-Eulerian coordinate}).
\item la troisième tâche concerne enfin la fourniture d'un produit numérique élaboré sur la base des maquettes numériques du détroit de Gibraltar telles que mises en place à la tâche 1.
\end{itemize}

\label{simuFE}
\section{Simulation des "fines échelles" \textit{(Groupe de recherche CROCO-Aérologie) }}

\noindent \textit{\textbf{Dynamique des fines échelles}}\\
Le groupe de modélisation océanique du Laboratoire d'Aérologie\footnote{Laboratoire de l'Observatoire Midi-Pyrénées}, contractant du projet Gibraltar 16CR01 et devenu en 2018 ''groupe CROCO-Aérologie``, étudie la \textit{dynamique des fines échelles océaniques} depuis maintenant une dizaine d'années. La qualification de \textit{fines échelles} englobe ici de façon très large une gamme de processus présentant une \textit{extension horizontale plus fine que le premier rayon de Rossby}. Cette définition couvre par conséquent des mécanismes et des processus dynamiques très hétérogènes:
\begin{itemize}
 \item les ondes de gravité (ondes de surface ou ondes internes) ou encore les ondes acoustiques ainsi que les mécanismes de génération et de déferlement associés à ces différents types d'ondes,
 \item les instabilités primaires de Kelvin-Helmholtz, les instabilités convectives... ainsi que la cascade turbulente directe initiée par ces instabilités primaires et conduisant au mélange diapycnal,
 \item les processus localisés liés au contrôle hydraulique des masses d'eau (ressauts hydrauliques...)
\end{itemize}
En parallèle, un projet pédagogique (SiGeoS) dédié à la turbulence géophysique a été mis en place aux niveaux Licence et Master par le groupe CROCO-Aérologie\footnote{SigeoS est un projet IDEX-Toulouse \textit{Pédagogie Innovante} proposant enseignement théorique et enseignement pratique en laboratoire et numérique. Le volet en Laboratoire du projet pédagogique SiGeoS est une collaboration avec le CNRM (Météo-France/CNRS)}. \\

\noindent \textit{\textbf{Simulation des grandes échelles turbulentes (LES)}}\\
Pour mener à bien ce projet de recherche et son alter-ego pédagogique, le groupe CROCO-Aérologie a été amené à faire évoluer ses outils numériques et en particulier son code numérique d'océan. La simulation des ''fines échelles océaniques`` nécessite en effet la mise en place d'une ''simulation des grandes structures turbulentes`` (en anglais \textit{LES}, pour Large Eddy Simulation). Les codes océaniques à toit libre (explicite) et pas-de-temps séparés (\textit{time-splitting} en anglais) devenus des classiques en modélisation océanique côtière et régionale au milieu des années 90 s'appuient tous sur l'hypothèse hydrostatique et ne permettent donc pas la simulation numérique tri-dimensionnelle des structures en vorticité comme le requière la LES. Le groupe CROCO-Aérologie a donc développé un code à toit libre et pas-de-temps séparés s'appuyant de façon originale sur un coeur non-hydrostatique (NH). Un algorithme par correction de pression (\textit{SNH}) a tout d'abord été implémenté dans le code Symphonie (Auclair et al., 2011). L'association de cet algorithme NH avec un schéma temporel à pas-de-temps séparés se révélant inappropriée\footnote{Association correction de pression / schéma temporel à pas-de-temps séparés: pour être efficace la correction de pression doit être implémentée dans le mode interne (lent), l'intégration et la correction du mode externe est numériquement mal posé et entraîne la génération de modes numériques potentiellement instables. D'un point de vue physique, les ondes de surface NH (ondes courtes) présentent en effet un cisaillement vertical de vitesse résolu par le seul mode lent (3D): le rapport des pas-de-temps des deux modes doit donc être proche de 1...}, un algorithme original, cette fois compressible (dit \textit{''non Boussinesq``}) a alors été proposé et implémenté en mode recherche dans le même code Symphonie (\textit{SNBQ}) (Auclair et al., 2018). \\

\noindent \textit{\textbf{Non-hydrostaticité et performances numériques}}\\
Ces développements algorithmiques laissant entrevoir des perspectives prometteuses pour l'étude des processus dynamiques, la nécessité de réaliser d'importants gains de performances a alors conduit à une remise en cause globale des schémas temporels, des schémas d'advection d'ordre réduit et des schémas de turbulence anisotropes généralement utilisés en océanographie régionale et côtière et plus spécifiquement dans \textit{SNBQ}. L'algorithme non-hydrostatique, non-Boussinesq développé en mode recherche dans \textit{SNBQ} a donc été implémenté dans ROMS-AGRIF rebaptisé CROCO (Auclair et al., \textit{publication en cours}). \\
Des gains de performances importants (à minima un ordre de grandeur) ont ainsi pu être réalisés grâce tout d'abord à plusieurs spécificités du code ROMS (Shchepetkin et McWilliams, 2005): 
\begin{itemize}
	\item son écriture et sa structure épurée facilitant la gestion de la \textit{mémoire-cache} des processeurs, 
	\item ou encore son implantation MPI particulièrement efficace. 
\end{itemize}

\noindent Une réécriture et une restructuration complète de l'algorithme non-hydrostatique, non-Boussinesq en un coeur numérique bi-modal\footnote{\textit{SNBQ} présente un coeur numérique à trois modes (modes interne(3D)/externe(2D)/NBQ(3D) alors que CROCO-NBQ est bi-modal (modes lent et rapide (NBQ) tous deux 3D.} ont finalement permis d'atteindre un niveau de performance numérique compatible avec la simulation \textit{LES} de la région du détroit de Gibraltar.\\
Le développement, les gains de performances et la consolidation du cœur numérique CROCO-NBQ ont pu être menés à bien grâce aux compétences complémentaires des instituts et universités associés au GdR CROCO (\textit{SHOM, IFREMER, INRIA, IRD}, \textit{INSU-CNRS} et \textit{Université de Toulouse Paul Sabatier}). Le code CROCO, base de la présente étude, est un véritable outil \textit{communautaire} de part son développement, son utilisation et sa distribution.\\


\section{Le détroit de Gibraltar: un "démonstrateur"}

En plus de ses enjeux stratégiques, le détroit de Gibraltar présente un indéniable intérêt en termes de dynamique océanique. A l'échelle de la circulation générale, il est l'unique lieu d'échange entre les bassins méditerranéen et Nord-Atlantique. La bathymétrie peu profonde du détroit impose un contrôle hydraulique sur ces échanges entraînant ressauts hydrauliques, générations d'ondes internes de grande amplitude, instabilités en cascade... autant de mécanismes qui induisent in fine un mélange intense des deux masses d'eau.\\
La qualité des prévisions de la circulation générale et des principales masses d'eau en Atlantique Nord et en Méditerranée dépend donc pour partie de notre capacité à comprendre et à simuler les principaux mécanismes et processus dynamiques de \textit{fine échelle} à l'œuvre dans le détroit de Gibraltar.\\

\noindent \textbf{\textit{Expertise dynamique}}\\
Lors du lancement du projet Gibraltar 16CR01 en 2016, le groupe CROCO-Aérologie avait déjà consacré une première étude à la région du détroit de Gibraltar avec la thèse de Lucie Bordois (Bordois, 2015, Bordois et al. 2016, Bordois et al., 2017). Cette étude avait permis :
\begin{itemize}
	\item  d'évaluer la "faisabilité" d'une simulation numérique des \textit{fines échelles} dans le détroit. Une maquette reposant sur une section verticale 2D avec les codes de recherche SNH puis SNBQ a ainsi été développée, les coûts de calcul et la qualité de la dynamique simulée ont été évalués.
	\item de réaliser une analyse approfondie des régimes dynamiques induits par la présence d'un détroit.
	\item de mettre en évidence la génération d'ondes internes de grande amplitude de Mode 1 et 2 au dessus des seuils de Camarinal et d'Espartel.
\end{itemize}
La thèse de Lucie Bordois a aussi confirmé la nécessité de franchir un cap en matière d'efficacité numérique: le code de recherche SNBQ se devait d'évoluer vers des méthodologies "massivement parallèles" efficaces. Une approche communautaire apparaissait alors clairement comme la seule alternative pouvant rapidement ouvrir les portes de la LES en océanographie. \\

\noindent \textbf{\textit{L'efficacité numérique: un enjeu communautaire}}\\
La région du détroit a par conséquent été choisie par le GdR CROCO comme "zone test" \textit{(démonstrateur}) et le groupe CROCO-Aérologie a concentré sur cette région l'essentiel de ses efforts de développements du cœur numérique non-hydrostatique et non-Boussinesq \textit{CROCO-NBQ} avec plusieurs objectifs:
\begin{itemize}
 \item \textit{mieux comprendre et mieux décrire la dynamique de fine échelle dans la région du détroit}. Les mécanismes de génération des ressauts hydrauliques, des ondes stationnaires orographiques \footnote{\textit{"Ocean lee waves"} en anglais.}, des ondes internes de grande amplitude (ondes solitaires) et des instabilités primaires     constituent autant de clés pour aborder la difficile problématique du mélange des masses d'eau (dans cette région et plus généralement dans l'océan global).
 \item  \textit{mettre en place par étape une maquette LES du détroit}. Cette maquette doit permettre à minima la simulation explicite des instabilités primaires se développant à l'interface des eaux méditerranéennes et atlantiques et doit in fine offrir une représentation réaliste et précise du mélange diapycnal intense entre ces deux masses d'eaux.
 \item \textit{comprendre et simuler la formation du jet méditerranéen dans le détroit}. C'est dans la région du détroit que cette composante importante de la circulation dans le bassin Nord-Atlantique prend naissance et acquière ses principales caractéristiques (densité, vorticité potentielle...).
\end{itemize}

\section{Méthodologie et engagements contractuels}

\noindent\textit{\textbf{Une hiérarchie de maquettes}}\\
Le groupe CROCO-Aérologie a choisi une méthodologie de développement numérique et d'exploration de la dynamique s'appuyant sur une hiérarchie de \textit{maquettes numériques} de complexité et de réalisme croissants:
\begin{itemize}
\item raffinement par étape de la physique à partir de maquettes hydrostatiques puis non-hydrostatiques et non-Boussinesq,
\item raffinement d'échelles avec des résolutions spatiales de l'ordre de quelques centaines de mètres (220 m pour la maquette NH-REF) réduites dans un second temps à seulement quelques dizaines de mètres (45 m pour la maquette NH-HR).
\end{itemize}
Il est à noter que la résolution la plus basse proposée dans le cadre du présent rapport est déjà 5 fois plus fine que la résolution initialement proposée dans le cadre du contrat de recherche Gibraltar 16CR1. En effet, la "\textit{résolution dans une gamme équivalente à celles de la maquette HYCOM utilisée par le SHOM pour cette région}" (soit environ 1.8 km) proposée dans le contrat s'est révélée beaucoup trop grande pour explorer les \textit{fines échelles} dans la région du détroit. Les résolutions de l'ordre de 220 m (maquette NH-REF) et de 45 m (maquette NH-HR) permettent de représenter et d'explorer les ondes internes de grande amplitude \textit{(ondes solitaires)} et les instabilités dynamiques primaires se développant entre les deux masses d'eau et conduisant à l'intensification locale du mélange (pour la résolution la plus fine seulement). Le travail en amont sur les performances numériques du coeur CROCO-NBQ (Section \ref{simuFE}) a permis de réaliser ce saut en résolution.\\
\noinden Chaque nouvelle maquette a été implémentée et sa dynamique explorée en deux temps. Des sections verticales bi-dimensionnelles alignées avec le transect principal de la campagne Gibraltar Experiment (Farmer et Armi, 1988) ont tout d'abord été mises en place, leur dynamique étant dans la foulée détaillée. Cette stratégie de dégradation des maquettes 3D est un choix reposant sur deux objectifs principaux:
\begin{itemize}
\item réduire les coûts de calcul inhérents à la mise au point des nouveaux algorithmes et des nouveaux schémas numériques,
\item simplifier la dynamique océanique complexe en l'assimilant à celle d'un canal hypothétiquement aligné avec l'axe du détroit.
\end{itemize}

\noindent\textit{\textbf{Une dynamique simplifiée}}\\
Un choix méthodologique fort a ensuite consisté, pour chaque maquette, à ne retenir que les forçages et les caractéristiques hydrologiques indispensables à la mise en place de la dynamique à explorer:
\begin{itemize}
\item Identifiée comme un élément clé de la dynamique régionale, une bathymétrie précise de la région du détroit a été fournie par le SHOM.
\item Les ondes et les courants de marée sont imposés de façon réaliste aux frontières des différentes maquettes.
\item Les deux principales masses d'eau (d'origine atlantique et méditerranéenne) ont dans un premier temps été simulées via une stratégie dite en \textit{Lock-Exchange} avant d'être forcées directement à partir des prévisions du modèle de circulation à "haute résolution" MIT-GCM fournies par l'ENEA \footnote{http://www.enea.it/it/seguici/pubblicazioni/pdf-volumi/cresco-report-2016.pdf. Nous remercions G. Sannino et l'ENEA pour ces sorties de simulation ainsi que pour l'accueil à Rome (Italie) de Lucie Bordois, alors post-doctorante dans le cadre du présent contrat.}.
\item Les forçages atmosphériques et telluriques ont, en première approximation, été négligés.
\end{itemize}

%%%%%%%%%%%%%%%%%%%%%%%%%%%%%%%%%%%%%%%%%%%%%%%%%%%%%%%%%%%%%%%%%%%
%%%%%%%%%%%%%%%%%%%%%%%%%%%%%%%%%%%%%%%%%%%%%%%%%%%%%%%%%%%%%%%%%%%
% Chapitre: Modélisation des fines échelles
%%%%%%%%%%%%%%%%%%%%%%%%%%%%%%%%%%%%%%%%%%%%%%%%%%%%%%%%%%%%%%%%%%%
%%%%%%%%%%%%%%%%%%%%%%%%%%%%%%%%%%%%%%%%%%%%%%%%%%%%%%%%%%%%%%%%%%%
\chapter{Modélisation des fines échelles dans le détroit de Gibraltar\\}
\label{chapitremodele}

La Tâche 1 du contrat Gibraltar 16CR01 est scindée en deux sous-tâches portant respectivement sur:
\begin{itemize}
\item{le développement et la validation d’une hiérarchie de maquettes hydrostatiques et non-hydrostatiques de la région du détroit de Gibraltar \textit{(Tâche 1.1)}}
\item{la dynamique à haute résolution dans le détroit de Gibraltar \textit{(Tâche 1.2)}}
\end{itemize}

\noindent Les maquettes et la dynamique qu'elles ont respectivement permis d'explorer sont maintenant décrites par ordre croissant de complexité pour offrir un rendu graduel des résultats obtenus. L'ordre proposé correspond de plus à la chronologie de développement et d'exploration choisie. \\

La Tâche 2 a été confiée à Eric Chassignet, visiteur scientifique au laboratoire d'Aérologie dans le cadre du présent contrat. Elle a pour objectif d'évaluer la pertinence d'une évolution du système de coordonnées verticales du code communautaire CROCO.

%%%%%%%%%%%%%%%%%%%%%%%%%%%%%%%%%%%%%%%%%%%%%%%%%%%%%%%%%%%%%%%%%%%
%%%%%%%%%%%%%%%%%%%%%%%%%%%%%%%%%%%%%%%%%%%%%%%%%%%%%%%%%%%%%%%%%%%
%%% Sections verticales 2D
%%%%%%%%%%%%%%%%%%%%%%%%%%%%%%%%%%%%%%%%%%%%%%%%%%%%%%%%%%%%%%%%%%%
%%%%%%%%%%%%%%%%%%%%%%%%%%%%%%%%%%%%%%%%%%%%%%%%%%%%%%%%%%%%%%%%%%%
\section{Sections verticales 2D: maquettes NH-REF et NH-HR \textit{(Tâche 1.1)}}
\label{section2D}

\noindent\fbox{\noindent\begin{minipage}{1\textwidth}
Une configuration numérique simple 2D est implémentée dans le détroit de Gibraltar avec le code communautaire CROCO dans sa version non-hydrostatique, non-Boussinesq (CROCO-NBQ). Cette configuration est peu coûteuse en temps de calcul et incorpore la bathymétrie le long de l'axe du détroit avec sa configuration de seuils (Farmer et Armi, 1988, programme \textit{Gibraltar Experiment}). Dans l'élaboration de cette configuration, une attention toute particulière a été apportée au rôle de la pseudo-force de Coriolis sur l'échange moyen simulé lors de l'initialisation par \textit{lock-exchange}.\\
Malgré les défauts inhérents à une représentation 2D-verticale (nécessairement limitée dynamiquement et non représentative des effets longitudinaux comme le contrôle dans le détroit de Tarifa), la configuration proposée permet de représenter de façon réaliste les mécanismes de \textit{fine échelle} dans le détroit à la fréquence de la marée barotrope : propagation des deux premiers modes d'ondes internes, contrôles hydrauliques aux seuils, ressaut hydraulique, et mélange turbulent. En particulier, les ondes internes de grande amplitude de mode 1 se propageant dans l'est du détroit sont caractérisées comme ondes solitaires (ou solitons) par comparaison avec le modèle analytique non-linéaire de Korteweig de Vries.\\
La modulation des phénomènes observés par divers paramètres (bathymétrie, intensité des courants de marée, hypothèse hydrostatique, résolution spatiale) est étudiée en détail. A haute résolution (environ 45 m), la relaxation de l'hypothèse non-hydrostatique est indispensable pour représenter les instabilités de cisaillement qui apparaissent dans le jet Méditerranéen, et qui constituent l'amorce de la cascade turbulente directe.
\end{minipage}
}

\bigskip\bigskip
\noindent Les résultats présentés dans la section (\ref{chapitremodele}.\ref{section2D}) donnent lieu à la rédaction d'un manuscrit. Margaux Hilt aidée par l'équipe de développement du GdR CROCO est en charge de la soumission prochaine de ce manuscrit pour publication dans un journal international de rang A. Le reste de la section est donc proposé en anglais.

%%%%%%%%%%%%%%%%%%%%%%%%%%%%%%%%%%%%%%%%%%%%%%%%%%%%%%%%%%%
%%%debut article 2D MH
%%%%%%%%%%%%%%%%%%%%%%%%%%%%%%%%%%%%%%%%%%%%%%%%%%%%%%%%%%%%%%%%%%%
\selectlanguage{english}
\input{./paper_gbr2d-content.tex} 
\selectlanguage{french}

\newpage

%%%%%%%%%%%%%%%%%%%%%%%%%%%%%%%%%%%%%%%%%%%%%%%%%%%%%%%%%%%%%%%%%%%
%%%%%%%%%%%%%%%%%%%%%%%%%%%%%%%%%%%%%%%%%%%%%%%%%%%%%%%%%%%%%%%%%%%
%%% Maquettes 3D NH-HR
%%%%%%%%%%%%%%%%%%%%%%%%%%%%%%%%%%%%%%%%%%%%%%%%%%%%%%%%%%%%%%%%%%%
%%%%%%%%%%%%%%%%%%%%%%%%%%%%%%%%%%%%%%%%%%%%%%%%%%%%%%%%%%%%%%%%%%%
\section{Maquette 3D NH-REF \textit{(Tâche 1.1)}}
\label{section3DNHREF}

\noindent\fbox{\noindent\begin{minipage}{1\textwidth}
\noindent La maquette tri-dimensionnelle NH-REF est, comme son nom l'indique, la maquette de référence implémentée et exploitée dans le cadre du présent contrat. Cette maquette s'appuie sur le coeur non-hydrostatique et non-Boussinesq CROCO et sur des schémas d'advection d'ordre élevé. Elle a permis de mettre en place une première simulation des "fines échelles" pour un faible coût de calcul. La dynamique 3D ainsi simulée a confirmé les caractéristiques des processus et autres mécanismes mis en évidence par Bordois (2015) à partir d'une section verticale le long du transect principal de la campagne \textit{"Gibraltar Experiment"}. Une évaluation de la pertinence de la marée barotrope simulée a plus particulièrement été réalisée. Les caractéristiques du ressaut hydraulique et des trains d'ondes solitaires ont dans un second temps été confrontées aux observations disponibles dans la région du détroit. La variabilité de ces processus à l'amplitude de la marée a été étudiée.
\end{minipage}
}

\bigskip\bigskip
\noindent Les résultats présentés dans la section (\ref{chapitremodele}.\ref{section3DNHREF}) donnent lieu à la rédaction d'un manuscrit. Lucie Bordois aidée par l'équipe de développement du GdR CROCO est en charge de la soumission prochaine de ce manuscrit pour publication dans un journal international de rang A. Le reste de la section est donc proposé en anglais.
%%%%%%%%%%%%%%%%%%%%%%%%%%%%%%%%%%%%%%%%%%%%%%%%%%%%%%%%%%%%%%%%%%%
%%%debut article NH-REF Lucie
%%%%%%%%%%%%%%%%%%%%%%%%%%%%%%%%%%%%%%%%%%%%%%%%%%%%%%%%%%%%%%%%%%%
\selectlanguage{english}
%%%%%%%%%%%%  Generated using docx2latex.com  %%%%%%%%%%%%%%

%%%%%%%%%%%%  v2.0.0-beta  %%%%%%%%%%%%%%

%\documentclass[12pt]{report}
%\usepackage{amsmath}
%\usepackage{latexsym}
%\usepackage{amsfonts}
%\usepackage[normalem]{ulem}
%\usepackage{array}
%\usepackage{amssymb}
%\usepackage{graphicx}
%\usepackage[backend=biber,
%style=numeric,
%sorting=none,
%isbn=false,
%doi=false,
%url=false,
%]{biblatex}\addbibresource{bibliography.bib}

%\usepackage{subfig}
%\usepackage{wrapfig}
%\usepackage{wasysym}
%\usepackage{enumitem}
%\usepackage{adjustbox}
%\usepackage{ragged2e}
%\usepackage[svgnames,table]{xcolor}
%\usepackage{tikz}
%\usepackage{longtable}
%\usepackage{changepage}
%\usepackage{setspace}
%\usepackage{hhline}
%\usepackage{multicol}
%\usepackage{tabto}
%\usepackage{float}
%\usepackage{multirow}
%\usepackage{makecell}
%\usepackage{fancyhdr}
%\usepackage[toc,page]{appendix}
%\usepackage[hidelinks]{hyperref}
%\usetikzlibrary{shapes.symbols,shapes.geometric,shadows,arrows.meta}
%\tikzset{>={Latex[width=1.5mm,length=2mm]}}
%\usepackage{flowchart}\usepackage[paperheight=11.69in,paperwidth=8.27in,left=0.98in,right=0.98in,top=0.98in,bottom=0.98in,headheight=1in]{geometry}
%\usepackage[utf8]{inputenc}
%\usepackage[T1]{fontenc}
%\TabPositions{0.5in,1.0in,1.5in,2.0in,2.5in,3.0in,3.5in,4.0in,4.5in,5.0in,5.5in,6.0in,}

%\urlstyle{same}


 %%%%%%%%%%%%  Set Depths for Sections  %%%%%%%%%%%%%%

% 1) Section
% 1.1) SubSection
% 1.1.1) SubSubSection
% 1.1.1.1) Paragraph
% 1.1.1.1.1) Subparagraph


%\setcounter{tocdepth}{5}
%\setcounter{secnumdepth}{5}


 %%%%%%%%%%%%  Set Depths for Nested Lists created by \begin{enumerate}  %%%%%%%%%%%%%%

\setlistdepth{9}
\renewlist{enumerate}{enumerate}{9}
		\setlist[enumerate,1]{label=\arabic*)}
		\setlist[enumerate,2]{label=\alph*)}
		\setlist[enumerate,3]{label=(\roman*)}
		\setlist[enumerate,4]{label=(\arabic*)}
		\setlist[enumerate,5]{label=(\Alph*)}
		\setlist[enumerate,6]{label=(\Roman*)}
		\setlist[enumerate,7]{label=\arabic*}
		\setlist[enumerate,8]{label=\alph*}
		\setlist[enumerate,9]{label=\roman*}

\renewlist{itemize}{itemize}{9}
		\setlist[itemize]{label=$\cdot$}
		\setlist[itemize,1]{label=\textbullet}
		\setlist[itemize,2]{label=$\circ$}
		\setlist[itemize,3]{label=$\ast$}
		\setlist[itemize,4]{label=$\dagger$}
		\setlist[itemize,5]{label=$\triangleright$}
		\setlist[itemize,6]{label=$\bigstar$}
		\setlist[itemize,7]{label=$\blacklozenge$}
		\setlist[itemize,8]{label=$\prime$}

 %%%%%%%%%%%%  Header here  %%%%%%%%%%%%%%
%\pagestyle{fancy}
%\fancyhf{}
%\renewcommand{\headrulewidth}{0pt}
%\setlength{\topsep}{0pt}\setlength{\parindent}{0pt}
%\renewcommand{\arraystretch}{1.3}

%%%%%%%%%%%%%%%%%%%% Document code starts here %%%%%%%%%%%%%%%%%%%%

\vspace{\baselineskip}
\subsection{Introduction}

The Gibraltar strait system controls the exchange between the Mediterranean basin and the global ocean. In this region, large topographic variations and strong tidal currents (up to 1.8 m/s above Camarinal sill) lead to complex generation mechanisms of energetic non-linear internal waves (Vázquez et al., 2006; Vlasenko et al., 2009). The propagation of these waves is highly influenced by the complex basin geometry (Sánchez-Garrido et al., 2011) and the hydraulic criticality of the flow (Sannino, 2009; Sannino et al., 2007). These local fine-scale processes are driving turbulence levels in the order of magnitude larger than open-ocean levels (Wesson and Gregg, 1994) possibly affecting water-mass exchange (Sannino et al., 2015). The interplays of these processes with the vertical mixing and local circulation are still an ongoing scientific issue. A non-hydrostatic and non-Boussinesq model (the CROCO ocean community model) has been implemented in a realistic 3D configuration on the strait of Gibraltar area to model explicitly these fine scale processes. The 3D high resolution (220 m) model is forced by the main barotropic tidal components (i.e. M2, S2, K1, O1) through the specification of the ENEA Mediterranean-Black Sea tidal operative forecasting system solutions. It provides a good representation of the barotropic $\&$  baroclinic tides in the strait of Gibraltar, representing well the internal fine-scale activity: hydraulic jumps formation above the main sills, internal solitary wave’s generation $\&$ propagation, neap-spring tide variability... \par

\vspace{\baselineskip}
\subsection{Modeling approach}

\vspace{\baselineskip}
\subsubsection{Non-Hydrostatic, non-Boussinesq Kernel of CROCO community code}

\vspace{\baselineskip}
Simulations are performed using the free-surface non-hydrostatic and non-Boussinesq kernel of the coastal and regional ocean community model CROCO. CROCO is a new oceanic modeling system built upon ROMS-AGRIF and the non-hydrostatic kernel of SNH (Auclair et al., 2018). It solves the dynamic and thermodynamic equations using stretched, terrain-following vertical s-coordinates and orthogonal, curvilinear horizontal coordinates with a two-mode time-splitting between baroclinic and barotropic modes. For general circulation purposes, it is forced by atmospheric momentum, heat and tracer fluxes. For a more extensive description and technical details of the hydrodynamic model as well as earlier applications to the US West Coast, we refer to Shchepetkin and McWilliams (1998, 2003, 2005), Marchesiello et al. (2001, 2003) and to Penven\ et al. (2006).  \par

\vspace{\baselineskip}
The nonhydrostatic and non-Boussinesq version of CROCO uses an original two-mode time-splitting technique between barotropic (\textit{$ \Delta $ t}{\fontsize{7pt}{8.4pt}\selectfont \textit{e}}) and non-hydrostatic (\textit{$ \Delta t$}{\fontsize{7pt}{8.4pt}\selectfont \textit{NBQ}}) motions to deal with free surface, non-hydrostatic flows and acoustic waves which are explicitly simulated. In the present non-Boussinesq\ version, a time-splitting is performed to circumvent the drastic CFL criteria induced by the high phase-velocity of such waves, $c_s$. Acoustic waves or more exactly $``$pseudo-acoustic$"$\footnote{When necessary and when pertinent, acoustic-wave velocity can indeed be lowered to reduce computing costs.}  waves have indeed been re-introduced to reduce computational costs. Indeed, this avoids Boussinesq-mathematical degeneracy which inevitably requires the inversion of a 3D Poisson-system in non-hydrostatic pressure-correction methods. As long as they remain faster than the fastest physical processes in the domain, the so-called $``$pseudo-acoustic$"$ wave phase-velocity can artificially be slowed down rendering unphysical high-frequency processes associated with bulk compressibility but preserving a coherent "slow", non-hydrostatic dynamics with a softening of the CFL criterion.\par

\vspace{\baselineskip}
To model explicitly internal fine-scale turbulent processes in Gibraltar strait,  a Monotone-Integrated Large Eddy Simulation{\fontsize{11pt}{13.2pt}\selectfont  }(MILES) approach has been chosen. In MILES, the dissipative nature of some classes of advection schemes is exploited to provide an implicit model of turbulence. Small, molecular values of the kinematic viscosity ($\nu\ = 10^{-6} m^{2}/s$) and density diffusivity ($K_\rho\ = 10^{-6} m^2/s$) are used, without neither turbulence closure scheme nor additional viscosity. Numerical dissipation is provided through a Total Variation Diminishing (TVD) advection scheme for the momentum and a fifth-order WENO advection-scheme for tracers. TVD and WENO-5 schemes are modern shock-capturing schemes which allow the simulation of sharp discontinuities. The objective is obviously not to reproduce the entire turbulent cascade but to be able to resolve explicitly the largest features of the internal fines-scale turbulent processes. \par


\vspace{\baselineskip}
\subsubsection{Numerical configurations}

\vspace{\baselineskip}
A non-hydrostatic (R\textsubscript{NBQ}) and a hydrostatic (R\textsubscript{H}) simulation are performed. A 3D version of the model is used with lateral Orlanski radiation conditions and free surface boundary conditions. A constant quadratic bottom stress formulation is used with a drag coefficient of $10^{-3}$. The model domain extends from the Gulf of Cadiz to the Alboran sea (Figure \ref{Fig1_Lucie}). The horizontal grid is constant with a resolution of 220 m and the vertical model grid is based on 40 $\sigma$-levels. Initial conditions for temperature and salinity are given by the ENEA Mediterranean-Black Sea tidal operative forecasting system\footnote{http://www.enea.it/it/seguici/pubblicazioni/pdf-volumi/cresco-report-2016.pdf}. Wind and surface fluxes are not included in the simulations. Tides are incorporated through the specification at the lateral boundary for the barotropic velocity and the surface elevation. Both are given by the ENEA Mediterranean-Black Sea tidal operative forecasting system. The ENEA forecasting system provides also lateral boundary conditions for salinity, temperature and baroclinic velocities (geostrophic currents). Table \ref{Tab1_Lucie} summarizes the numerical and physical parameters of both configurations. \par

%%%%%%%%%%%%%%%%%%%% Table No: 1 starts here %%%%%%%%%%%%%%%%%%%%

\begin{table}[!h]
 			\centering
\begin{tabular}{p{1.56in}p{4.48in}}
\hline

%row no:2
\multicolumn{1}{|p{1.56in}}{(Nx, Ny, Nz, Nt) } & 
\multicolumn{1}{|p{4.48in}|}{\Centering (408, 523, 40, 669 600)} \\
\hhline{--}
%row no:3
\multicolumn{1}{|p{1.56in}}{Simulated period} & 
\multicolumn{1}{|p{4.48in}|}{\Centering [10/09/2017 – 10/10/2017]} \\
\hhline{--}
%row no:4
\multicolumn{1}{|p{1.56in}}{($ \Delta $ x, $ \Delta $ y, $ \Delta $ z) } & 
\multicolumn{1}{|p{4.48in}|}{\Centering (220 m, 220 m, 7-23 m)} \\
\hhline{--}
%row no:5
\multicolumn{1}{|p{1.56in}}{($ \Delta $ t, N\textsubscript{e}) } & 
\multicolumn{1}{|p{4.48in}|}{\Centering (4s, 9)} \\
\hhline{--}
%row no:6
\multicolumn{1}{|p{1.56in}}{Cs (for R\textsubscript{NBQ}) } & 
\multicolumn{1}{|p{4.48in}|}{\Centering 400 m/s} \\
\hhline{--}
%row no:7
\multicolumn{1}{|p{1.56in}}{Bathymetry } & 
\multicolumn{1}{|p{4.48in}|}{\Centering SHOM (500 m)} \\
\hhline{--}
%row no:8
\multicolumn{1}{|p{1.56in}}{\multirow{1}{*}{\begin{tabular}{p{1.56in}}Tidal forcing \\\end{tabular}}} & 
\multicolumn{1}{|p{4.48in}|}{\Centering ENEA Mediterranean-Black Sea tidal operative forecasting system} \\
\hhline{~-}
%row no:9
\multicolumn{1}{|p{1.56in}}{} & 
\multicolumn{1}{|p{4.48in}|}{\Centering M2, S2, K1, O1} \\
\hhline{--}
%row no:10
\multicolumn{1}{|p{1.56in}}{Stratification } & 
\multicolumn{1}{|p{4.48in}|}{\Centering ENEA Mediterranean-Black Sea tidal operative forecasting system} \\
\hhline{--}
%row no:11
\multicolumn{1}{|p{1.56in}}{Geostrophic currents } & 
\multicolumn{1}{|p{4.48in}|}{\Centering ENEA Mediterranean-Black Sea tidal operative forecasting system} \\
\hhline{--}
%row no:12
\multicolumn{1}{|p{1.56in}}{Atmospheric forcing } & 
\multicolumn{1}{|p{4.48in}|}{\Centering No} \\
\hhline{--}
%row no:13
\multicolumn{1}{|p{1.56in}}{Advection schemes } & 
\multicolumn{1}{|p{4.48in}|}{\Centering TVD (U-V-W) $\&$  WENO5 (T-S)} \\
\hhline{--}

\end{tabular}
%%%%%%%%%%%%%%%%%%%% Table No: 1 ends here %%%%%%%%%%%%%%%%%%%%

\caption{Numerical $\&$  physical parameters of \textbf{R\textsubscript{H}} and \textbf{R\textsubscript{NBQ}}}
\label{Tab1_Lucie}
\end{table}

\vspace{\baselineskip}
The bathymetry is extracted from a 500-m-resolution topographic data set of the Strait of Gibraltar provided from the SHOM (HOMONIM 500-m-Grid). A Gaussian interpolation of the topography data has been made\footnote{Parameters of the Gaussian interpolation: Gaussian radius: $r_{max}\ =\ 0.25$, minimum depth $h_{min}\ =\ 30\ m$.} to set up the 220-m-grid. The resulting model bathymetry is illustrated on figure \ref{Fig1_Lucie}.a. On the vertical section of figure \ref{Fig1_Lucie}.b, the dominant topographic features of the strait are clearly recognizable (from west to east): Espartel sill (ES); Tangier basin (TB); Camarinal sill (CS), with the minimum depth of 188 m, and Tarifa Narrows (TN) with the maximum depth of 960 m. This vertical section (located by the black line on figure \ref{Fig1_Lucie}.a) has been selected to incorporate ES, the shallowest point of CS and the deepest point of TN. The vertical resolution $(\Delta z\ =\ 7-23\ m$) is much higher above Camarinal sill than in the eastern and western ends. \par

\vspace{\baselineskip}

%%%%%%%%%%%%%%%%%%%% Figure/Image No: 1 starts here %%%%%%%%%%%%%%%%%%%%

\begin{figure}[!h]
	\begin{Center}
		\includegraphics[width=3.77in,height=4.44in]{./media/image1.jpg}
	\end{Center}

%%%%%%%%%%%%%%%%%%%% Figure/Image No: 1 Ends here %%%%%%%%%%%%%%%%%%%%

\caption { Bathymetry and domain extension of R\textsubscript{NBQ} $\&$  R\textsubscript{H}. The black line locate the vertical transect/section on the lower plot. Red squares locate tidal gauges position and green squares locate Xtrack position.}
 \label{Fig1_Lucie}
\end{figure}

\vspace{\baselineskip}
\subsection{Barotropic tides}

\vspace{\baselineskip}
Barotropic tides in the region present an original pattern due to the complex geometry of the strait and its strong dynamics. The tides are typically semi-diurnal, as in the whole Northern Atlantic Ocean, and flowing back and forth zonally at a first glance. The large tidal flows at the Atlantic entrance are strongly reduced in the strait to enter the Mediterranean Sea with typical amplitudes of a few tens of centimeters. The adjustment of the flow imposes complex tidal patterns into the Strait itself making the modelling of barotropic tides in the strait region very challenging. \\
This complexity is circumvented in state-of-the-art barotropic tidal models thanks to the assimilation of altimeter data particularly. But, into the Strait, the lack of data to constrain the cotidal charts makes these global models very precise outside the Strait but less inside. So the forcing they can impose at the boundary of a reduced modelling domain centered on the Strait is generally an important source of error. A huge modelling effort is thus necessary to assess the sensitivity of CROCO to various parameters (bathymetry, open-boundary conditions, forcing atlas...). Additional work (eg. bathymetry, bottom friction, large-scale forcing compatibility, ... ) will be needed to overcome the remaining difficulties.\\
Both configurations (R\textsubscript{NBQ} $\&$  R\textsubscript{H}) run over two spring neap cycles, and the results are compared with available observed data (X-TRACK altimetry and tidal gauges). \par
In Tables \ref{Tab1_Lucie}, \ref{Tab2_Lucie} and \ref{Tab3_Lucie} the observed and simulated amplitudes ($A$) and phases ($G$) are compared for the
main barotropic tidal components (M2, S2, K1, O1). As expected, the simulated amplitudes ($A$) and phases ($G$) of the four tidal components are very similar in the hydrostatic and non-hydrostatic runs.\\

A good agreement between observed and predicted values is found; the maximum differences do not exceed 4.8 cm in amplitude and 11.3° in phase for the semi-diurnal components. The relative errors (dE/A) are significant for the diurnal components (O1, K1), however the amplitudes of these components are quite small in that region.


%%%%%%%%%%%%%%%%%%%% Table No: 2 starts here %%%%%%%%%%%%%%%%%%%%


\begin{table}[!h]
 			\centering
\begin{tabular}{p{0.25in}p{0.31in}p{0.54in}p{0.54in}p{0.54in}p{0.5in}p{0.54in}p{0.54in}p{0.54in}p{0.5in}}
\hline
%row no:1
\multicolumn{2}{|p{0.56in}}{\Centering {\fontsize{11pt}{13.2pt}\selectfont Simulation}} & 
\multicolumn{4}{|p{2.12in}}{\Centering {\fontsize{11pt}{13.2pt}\selectfont R\textsubscript{NBQ}}} & 
\multicolumn{4}{|p{2.12in}|}{\Centering {\fontsize{11pt}{13.2pt}\selectfont R\textsubscript{H}}} \\
\hhline{----------}
%row no:2
\multicolumn{2}{|p{0.56in}}{\cellcolor[HTML]{EFEFEF}} & 
\multicolumn{1}{|p{0.54in}}{\cellcolor[HTML]{EFEFEF}\Centering {\fontsize{11pt}{13.2pt}\selectfont dA (cm)}} & 
\multicolumn{1}{|p{0.54in}}{\cellcolor[HTML]{EFEFEF}\Centering {\fontsize{11pt}{13.2pt}\selectfont dG ($ ^{\circ} $)}} & 
\multicolumn{1}{|p{0.54in}}{\cellcolor[HTML]{EFEFEF}\Centering {\fontsize{11pt}{13.2pt}\selectfont $ \vert\vert dE \vert\vert$  (cm)}} & 
\multicolumn{1}{|p{0.5in}}{\cellcolor[HTML]{EFEFEF}\Centering {\fontsize{11pt}{13.2pt}\selectfont \Tiny{\textbf{$ \vert\vert dE \vert\vert/A$ ($\%$ )}}}} & 
\multicolumn{1}{|p{0.54in}}{\cellcolor[HTML]{EFEFEF}\Centering {\fontsize{11pt}{13.2pt}\selectfont dA (cm)}} & 
\multicolumn{1}{|p{0.54in}}{\cellcolor[HTML]{EFEFEF}\Centering {\fontsize{11pt}{13.2pt}\selectfont dG ($ ^{\circ} $)}} & 
\multicolumn{1}{|p{0.54in}}{\cellcolor[HTML]{EFEFEF}\Centering {\fontsize{11pt}{13.2pt}\selectfont $ \vert\vert dE \vert\vert$ (cm)}} & 
\multicolumn{1}{|p{0.5in}|}{\cellcolor[HTML]{EFEFEF}\Centering {\fontsize{11pt}{13.2pt}\selectfont \Tiny{\textbf{$ \vert\vert dE \vert\vert/A$ ($\%$ )}}}} \\
\hhline{----------}
%row no:3
\multicolumn{1}{|p{0.25in}}{\multirowcell{}{}{\begin{tabular}{p{0.16in}}\Centering {\fontsize{11pt}{13.2pt}\selectfont M2}\\\end{tabular}}} & 
\multicolumn{1}{|p{0.31in}}{\Centering {\fontsize{11pt}{13.2pt}\selectfont North}} & 
\multicolumn{1}{|p{0.54in}}{\Centering {\fontsize{9pt}{10.8pt}\selectfont 4.8+-0.7}} & 
\multicolumn{1}{|p{0.54in}}{\Centering {\fontsize{9pt}{10.8pt}\selectfont 0.7+-3.6}} & 
\multicolumn{1}{|p{0.54in}}{\Centering {\fontsize{9pt}{10.8pt}\selectfont 3.4+-1.2}} & 
\multicolumn{1}{|p{0.5in}}{\Centering {\fontsize{10pt}{12.0pt}\selectfont \textbf{15.3}}} & 
\multicolumn{1}{|p{0.54in}}{\Centering {\fontsize{9pt}{10.8pt}\selectfont 4.8+-0.7}} & 
\multicolumn{1}{|p{0.54in}}{\Centering {\fontsize{9pt}{10.8pt}\selectfont 0.3+-3.6}} & 
\multicolumn{1}{|p{0.54in}}{\Centering {\fontsize{9pt}{10.8pt}\selectfont 3.4+-1.2}} & 
\multicolumn{1}{|p{0.5in}|}{\Centering {\fontsize{10pt}{12.0pt}\selectfont \textbf{15.3}}} \\
\hhline{~---------}
%row no:4
\multicolumn{1}{|p{0.25in}}{} & 
\multicolumn{1}{|p{0.31in}}{\Centering {\fontsize{11pt}{13.2pt}\selectfont South}} & 
\multicolumn{1}{|p{0.54in}}{\Centering {\fontsize{9pt}{10.8pt}\selectfont 4.5+-0.7}} & 
\multicolumn{1}{|p{0.54in}}{\Centering {\fontsize{9pt}{10.8pt}\selectfont -10.6+-4.4}} & 
\multicolumn{1}{|p{0.54in}}{\Centering {\fontsize{9pt}{10.8pt}\selectfont 4.3+-1.4}} & 
\multicolumn{1}{|p{0.5in}}{\Centering {\fontsize{10pt}{12.0pt}\selectfont \textbf{19.5}}} & 
\multicolumn{1}{|p{0.54in}}{\Centering {\fontsize{9pt}{10.8pt}\selectfont 4.4+-0.7}} & 
\multicolumn{1}{|p{0.54in}}{\Centering {\fontsize{9pt}{10.8pt}\selectfont -11.1+-4.4}} & 
\multicolumn{1}{|p{0.54in}}{\Centering {\fontsize{9pt}{10.8pt}\selectfont 4.4+-1.5}} & 
\multicolumn{1}{|p{0.5in}|}{\Centering {\fontsize{10pt}{12.0pt}\selectfont \textbf{19.6}}} \\
\hhline{----------}
%row no:5
\multicolumn{1}{|p{0.25in}}{\multirow{1}{*}{\begin{tabular}{p{0.16in}}\cellcolor[HTML]{EFEFEF}\Centering {\fontsize{11pt}{13.2pt}\selectfont S2}\\\end{tabular}}} & 
\multicolumn{1}{|p{0.31in}}{\cellcolor[HTML]{EFEFEF}\Centering {\fontsize{11pt}{13.2pt}\selectfont North}} & 
\multicolumn{1}{|p{0.54in}}{\cellcolor[HTML]{EFEFEF}\Centering {\fontsize{9pt}{10.8pt}\selectfont 0.1+-2}} & 
\multicolumn{1}{|p{0.54in}}{\cellcolor[HTML]{EFEFEF}\Centering {\fontsize{9pt}{10.8pt}\selectfont 11.3+-16}} & 
\multicolumn{1}{|p{0.54in}}{\cellcolor[HTML]{EFEFEF}\Centering {\fontsize{9pt}{10.8pt}\selectfont 0.9+-2}} & 
\multicolumn{1}{|p{0.5in}}{\cellcolor[HTML]{EFEFEF}\Centering {\fontsize{10pt}{12.0pt}\selectfont \textbf{24}}} & 
\multicolumn{1}{|p{0.54in}}{\cellcolor[HTML]{EFEFEF}\Centering {\fontsize{9pt}{10.8pt}\selectfont 0.2+-1.8}} & 
\multicolumn{1}{|p{0.54in}}{\cellcolor[HTML]{EFEFEF}\Centering {\fontsize{9pt}{10.8pt}\selectfont 11.3+-16}} & 
\multicolumn{1}{|p{0.54in}}{\cellcolor[HTML]{EFEFEF}\Centering {\fontsize{9pt}{10.8pt}\selectfont 0.8+-1.9}} & 
\multicolumn{1}{|p{0.5in}|}{\cellcolor[HTML]{EFEFEF}\Centering {\fontsize{10pt}{12.0pt}\selectfont \textbf{23.5}}} \\
\hhline{~---------}
%row no:6
\multicolumn{1}{|p{0.25in}}{\cellcolor[HTML]{EFEFEF}} & 
\multicolumn{1}{|p{0.31in}}{\cellcolor[HTML]{EFEFEF}\Centering {\fontsize{11pt}{13.2pt}\selectfont South}} & 
\multicolumn{1}{|p{0.54in}}{\cellcolor[HTML]{EFEFEF}\Centering {\fontsize{9pt}{10.8pt}\selectfont 1.2+-0.9}} & 
\multicolumn{1}{|p{0.54in}}{\cellcolor[HTML]{EFEFEF}\Centering {\fontsize{9pt}{10.8pt}\selectfont -5.5+-3}} & 
\multicolumn{1}{|p{0.54in}}{\cellcolor[HTML]{EFEFEF}\Centering {\fontsize{9pt}{10.8pt}\selectfont 1+-0.7}} & 
\multicolumn{1}{|p{0.5in}}{\cellcolor[HTML]{EFEFEF}\Centering {\fontsize{10pt}{12.0pt}\selectfont \textbf{14.1}}} & 
\multicolumn{1}{|p{0.54in}}{\cellcolor[HTML]{EFEFEF}\Centering {\fontsize{9pt}{10.8pt}\selectfont 1.2+1.0}} & 
\multicolumn{1}{|p{0.54in}}{\cellcolor[HTML]{EFEFEF}\Centering {\fontsize{9pt}{10.8pt}\selectfont -6.0+-3}} & 
\multicolumn{1}{|p{0.54in}}{\cellcolor[HTML]{EFEFEF}\Centering {\fontsize{9pt}{10.8pt}\selectfont 1+-0.7}} & 
\multicolumn{1}{|p{0.5in}|}{\cellcolor[HTML]{EFEFEF}\Centering {\fontsize{10pt}{12.0pt}\selectfont \textbf{14.3}}} \\
\hhline{----------}
%row no:7
\multicolumn{1}{|p{0.25in}}{\multirow{1}{*}{\begin{tabular}{p{0.16in}}\Centering {\fontsize{11pt}{13.2pt}\selectfont K1}\\\end{tabular}}} & 
\multicolumn{1}{|p{0.31in}}{\Centering {\fontsize{11pt}{13.2pt}\selectfont North}} & 
\multicolumn{1}{|p{0.54in}}{\Centering {\fontsize{9pt}{10.8pt}\selectfont -0.01+-0.1}} & 
\multicolumn{1}{|p{0.54in}}{\Centering {\fontsize{9pt}{10.8pt}\selectfont 2.2+-1.8}} & 
\multicolumn{1}{|p{0.54in}}{\Centering {\fontsize{9pt}{10.8pt}\selectfont 0.1+-0.7}} & 
\multicolumn{1}{|p{0.5in}}{\Centering {\fontsize{10pt}{12.0pt}\selectfont \textbf{20.4}}} & 
\multicolumn{1}{|p{0.54in}}{\Centering {\fontsize{9pt}{10.8pt}\selectfont -0.02+-1.0}} & 
\multicolumn{1}{|p{0.54in}}{\Centering {\fontsize{9pt}{10.8pt}\selectfont 0.8+-1.1}} & 
\multicolumn{1}{|p{0.54in}}{\Centering {\fontsize{9pt}{10.8pt}\selectfont 0.1+-0.7}} & 
\multicolumn{1}{|p{0.5in}|}{\Centering {\fontsize{10pt}{12.0pt}\selectfont \textbf{21.1}}} \\
\hhline{~---------}
%row no:8
\multicolumn{1}{|p{0.25in}}{} & 
\multicolumn{1}{|p{0.31in}}{\Centering {\fontsize{11pt}{13.2pt}\selectfont South}} & 
\multicolumn{1}{|p{0.54in}}{\Centering {\fontsize{9pt}{10.8pt}\selectfont -1.7+-0.8}} & 
\multicolumn{1}{|p{0.54in}}{\Centering {\fontsize{9pt}{10.8pt}\selectfont 37.8+-2.5}} & 
\multicolumn{1}{|p{0.54in}}{\Centering {\fontsize{9pt}{10.8pt}\selectfont 1.6+-0.5}} & 
\multicolumn{1}{|p{0.5in}}{\Centering {\fontsize{10pt}{12.0pt}\selectfont \textbf{66.5}}} & 
\multicolumn{1}{|p{0.54in}}{\Centering {\fontsize{9pt}{10.8pt}\selectfont -1.5+-0.8}} & 
\multicolumn{1}{|p{0.54in}}{\Centering {\fontsize{9pt}{10.8pt}\selectfont 37.8+-2.3}} & 
\multicolumn{1}{|p{0.54in}}{\Centering {\fontsize{9pt}{10.8pt}\selectfont 1.6+-0.5}} & 
\multicolumn{1}{|p{0.5in}|}{\Centering {\fontsize{10pt}{12.0pt}\selectfont \textbf{65.4}}} \\
\hhline{----------}
%row no:9
\multicolumn{1}{|p{0.25in}}{\multirow{1}{*}{\begin{tabular}{p{0.16in}}\cellcolor[HTML]{EFEFEF}\Centering {\fontsize{11pt}{13.2pt}\selectfont O1}\\\end{tabular}}} & 
\multicolumn{1}{|p{0.31in}}{\cellcolor[HTML]{EFEFEF}\Centering {\fontsize{11pt}{13.2pt}\selectfont North}} & 
\multicolumn{1}{|p{0.54in}}{\cellcolor[HTML]{EFEFEF}\Centering {\fontsize{9pt}{10.8pt}\selectfont -0.8+-0.6}} & 
\multicolumn{1}{|p{0.54in}}{\cellcolor[HTML]{EFEFEF}\Centering {\fontsize{9pt}{10.8pt}\selectfont -10.7$ \pm $ 13.6}} & 
\multicolumn{1}{|p{0.54in}}{\cellcolor[HTML]{EFEFEF}\Centering {\fontsize{9pt}{10.8pt}\selectfont 0.7+-0.5}} & 
\multicolumn{1}{|p{0.5in}}{\cellcolor[HTML]{EFEFEF}\Centering {\fontsize{10pt}{12.0pt}\selectfont \textbf{46.6}}} & 
\multicolumn{1}{|p{0.54in}}{\cellcolor[HTML]{EFEFEF}\Centering {\fontsize{9pt}{10.8pt}\selectfont -1+-0.6}} & 
\multicolumn{1}{|p{0.54in}}{\cellcolor[HTML]{EFEFEF}\Centering {\fontsize{9pt}{10.8pt}\selectfont -8.1$ \pm $ 13.0}} & 
\multicolumn{1}{|p{0.54in}}{\cellcolor[HTML]{EFEFEF}\Centering {\fontsize{9pt}{10.8pt}\selectfont 0.8+-0.5}} & 
\multicolumn{1}{|p{0.5in}|}{\cellcolor[HTML]{EFEFEF}\Centering {\fontsize{10pt}{12.0pt}\selectfont \textbf{48.8}}} \\
\hhline{~---------}
%row no:10
\multicolumn{1}{|p{0.25in}}{\cellcolor[HTML]{EFEFEF}} & 
\multicolumn{1}{|p{0.31in}}{\cellcolor[HTML]{EFEFEF}\Centering {\fontsize{11pt}{13.2pt}\selectfont South}} & 
\multicolumn{1}{|p{0.54in}}{\cellcolor[HTML]{EFEFEF}\Centering {\fontsize{10pt}{12.0pt}\selectfont 0.2+-0.5}} & 
\multicolumn{1}{|p{0.54in}}{\cellcolor[HTML]{EFEFEF}\Centering {\fontsize{10pt}{12.0pt}\selectfont -8.5+-18.5}} & 
\multicolumn{1}{|p{0.54in}}{\cellcolor[HTML]{EFEFEF}\Centering {\fontsize{10pt}{12.0pt}\selectfont 0.2+-0.7}} & 
\multicolumn{1}{|p{0.5in}}{\cellcolor[HTML]{EFEFEF}\Centering {\fontsize{10pt}{12.0pt}\selectfont \textbf{25.6}}} & 
\multicolumn{1}{|p{0.54in}}{\cellcolor[HTML]{EFEFEF}\Centering {\fontsize{10pt}{12.0pt}\selectfont 0.09+-0.5}} & 
\multicolumn{1}{|p{0.54in}}{\cellcolor[HTML]{EFEFEF}\Centering {\fontsize{10pt}{12.0pt}\selectfont -5.7+-18.6}} & 
\multicolumn{1}{|p{0.54in}}{\cellcolor[HTML]{EFEFEF}\Centering {\fontsize{10pt}{12.0pt}\selectfont 0.1+-0.7}} & 
\multicolumn{1}{|p{0.5in}|}{\cellcolor[HTML]{EFEFEF}\Centering {\fontsize{10pt}{12.0pt}\selectfont \textbf{25.3}}} \\
\hhline{----------}

\end{tabular}


%%%%%%%%%%%%%%%%%%%% Table No: 2 ends here %%%%%%%%%%%%%%%%%%%%
\caption{Model performance statistics for the main barotropic tidal components (M2, S2, K1, O1) in \textbf{R\textsubscript{NBQ}}\textsubscript{ }$\&$ \textbf{ R\textsubscript{H}} in comparison to X-TRACK altimetry. dA = mean amplitude biais; dG = mean phase bias; $ \vert $ $ \vert $ dE$ \vert $ $ \vert $ \ = mean combined bias;  $ \vert $ $ \vert $ dE$ \vert $ $ \vert $ /A = mean relative bias.}
\label{Tab2_Lucie}

 \end{table}

%%%%%%%%%%%%%%%%%%%% Table No: 3 starts here %%%%%%%%%%%%%%%%%%%%


\begin{table}[!h]
 			\centering
\begin{tabular}{p{0.25in}p{0.4in}p{0.47in}p{0.6in}p{0.5in}p{0.49in}p{0.5in}p{0.5in}p{0.5in}p{0.5in}}
%\begin{tabular}{p{0.23in}p{0.4in}p{0.47in}p{0.61in}p{0.61in}p{0.49in}p{0.69in}p{0.59in}p{0.59in}p{0.69in}}
\hline
%row no:1
\multicolumn{2}{|p{0.65in}}{\Centering {\fontsize{11pt}{13.2pt}\selectfont Simulation}} & 
\multicolumn{4}{|p{2.06in}}{\Centering {\fontsize{11pt}{13.2pt}\selectfont Forcing (MITgcm)}} & 
\multicolumn{4}{|p{2.0in}|}{\Centering {\fontsize{11pt}{13.2pt}\selectfont TPX08}} \\
\hhline{----------}
%row no:2
\multicolumn{2}{|p{0.65in}}{\cellcolor[HTML]{EFEFEF}} & 
\multicolumn{1}{|p{0.47in}}{\cellcolor[HTML]{EFEFEF}\Centering {\fontsize{11pt}{13.2pt}\selectfont dA (cm)}} & 
\multicolumn{1}{|p{0.61in}}{\cellcolor[HTML]{EFEFEF}\Centering {\fontsize{11pt}{13.2pt}\selectfont dG ($ ^{\circ} $)}} & 
\multicolumn{1}{|p{0.61in}}{\cellcolor[HTML]{EFEFEF}\Centering {\fontsize{11pt}{13.2pt}\selectfont $ \vert\vert dE \vert\vert $  (cm)}} & 
\multicolumn{1}{|p{0.4in}}{\cellcolor[HTML]{EFEFEF}\Centering {\fontsize{11pt}{13.2pt}\selectfont \Tiny{\textbf{$ \vert\vert dE\vert\vert/A$ ($\%$)}}}} & 
\multicolumn{1}{|p{0.55in}}{\cellcolor[HTML]{EFEFEF}\Centering {\fontsize{11pt}{13.2pt}\selectfont dA (cm)}} & 
\multicolumn{1}{|p{0.5in}}{\cellcolor[HTML]{EFEFEF}\Centering {\fontsize{11pt}{13.2pt}\selectfont dG ($ ^{\circ} $)}} & 
\multicolumn{1}{|p{0.5in}}{\cellcolor[HTML]{EFEFEF}\Centering {\fontsize{11pt}{13.2pt}\selectfont $\vert\vert dE\vert\vert$ (cm)}} & 
\multicolumn{1}{|p{0.5in}|}{\cellcolor[HTML]{EFEFEF}\Centering {\fontsize{11pt}{13.2pt}\selectfont \Tiny{\textbf{$ \vert\vert dE \vert\vert/A$ ($\%$)}}}} \\
\hhline{----------}
%row no:3
\multicolumn{1}{|p{0.25in}}{\multirowcell{}{}{\begin{tabular}{p{0.01in}}\Centering {\fontsize{11pt}{13.2pt}\selectfont M2}\\\end{tabular}}} & 
\multicolumn{1}{|p{0.4in}}{\Centering {\fontsize{11pt}{13.2pt}\selectfont North}} & 
\multicolumn{1}{|p{0.47in}}{\Centering {\fontsize{10pt}{12.0pt}\selectfont 4.8 $ \pm $ 0.6}} & 
\multicolumn{1}{|p{0.61in}}{\Centering {\fontsize{10pt}{12.0pt}\selectfont 9.1 $ \pm $ 3.3}} & 
\multicolumn{1}{|p{0.61in}}{\Centering {\fontsize{10pt}{12.0pt}\selectfont 4.2$ \pm $ 1.1}} & 
\multicolumn{1}{|p{0.49in}}{\Centering {\fontsize{10pt}{12.0pt}\selectfont \textbf{18.5}}} & 
\multicolumn{1}{|p{0.5in}}{\Centering {\fontsize{10pt}{12.0pt}\selectfont 0.9 $ \pm $ 0.4}} & 
\multicolumn{1}{|p{0.5in}}{\Centering {\fontsize{10pt}{12.0pt}\selectfont 0.7$ \pm $ 3.6}} & 
\multicolumn{1}{|p{0.5in}}{\Centering {\fontsize{10pt}{12.0pt}\selectfont 0.6$ \pm $ 1.2}} & 
\multicolumn{1}{|p{0.5in}|}{\Centering {\fontsize{10pt}{12.0pt}\selectfont \textbf{4.8}}} \\
\hhline{~---------}
%row no:4
\multicolumn{1}{|p{0.25in}}{} & 
\multicolumn{1}{|p{0.4in}}{\Centering {\fontsize{11pt}{13.2pt}\selectfont South}} & 
\multicolumn{1}{|p{0.47in}}{\Centering {\fontsize{10pt}{12.0pt}\selectfont 5.6$ \pm $ 0.7}} & 
\multicolumn{1}{|p{0.61in}}{\Centering {\fontsize{10pt}{12.0pt}\selectfont 0.5$ \pm $ 4.7}} & 
\multicolumn{1}{|p{0.61in}}{\Centering {\fontsize{10pt}{12.0pt}\selectfont 3.9$ \pm $ 1.5}} & 
\multicolumn{1}{|p{0.49in}}{\Centering {\fontsize{10pt}{12.0pt}\selectfont \textbf{19.0}}} & 
\multicolumn{1}{|p{0.5in}}{\Centering {\fontsize{10pt}{12.0pt}\selectfont 0.3$ \pm $ 0.9}} & 
\multicolumn{1}{|p{0.5in}}{\Centering {\fontsize{10pt}{12.0pt}\selectfont -3.6$ \pm $ 4.5}} & 
\multicolumn{1}{|p{0.5in}}{\Centering {\fontsize{10pt}{12.0pt}\selectfont 1.2$ \pm $ 1.5}} & 
\multicolumn{1}{|p{0.5in}|}{\Centering {\fontsize{10pt}{12.0pt}\selectfont \textbf{6.1}}} \\
\hhline{----------}
%row no:5
\multicolumn{1}{|p{0.25in}}{\multirow{1}{*}{\begin{tabular}{p{0.01in}}\cellcolor[HTML]{EFEFEF}\Centering {\fontsize{11pt}{13.2pt}\selectfont S2}\\\end{tabular}}} & 
\multicolumn{1}{|p{0.4in}}{\cellcolor[HTML]{EFEFEF}\Centering {\fontsize{11pt}{13.2pt}\selectfont North}} & 
\multicolumn{1}{|p{0.47in}}{\Centering {\fontsize{10pt}{12.0pt}\selectfont 0.3$ \pm $ 1.8}} & 
\multicolumn{1}{|p{0.61in}}{\Centering {\fontsize{10pt}{12.0pt}\selectfont 17.5$ \pm $ 15.4}} & 
\multicolumn{1}{|p{0.61in}}{\Centering {\fontsize{10pt}{12.0pt}\selectfont 1.4$ \pm $ 1.9}} & 
\multicolumn{1}{|p{0.49in}}{\cellcolor[HTML]{EFEFEF}\Centering {\fontsize{10pt}{12.0pt}\selectfont \textbf{27.5}}} & 
\multicolumn{1}{|p{0.5in}}{\Centering {\fontsize{10pt}{12.0pt}\selectfont -1.0$ \pm $ 1.8}} & 
\multicolumn{1}{|p{0.5in}}{\Centering {\fontsize{10pt}{12.0pt}\selectfont 12.3$ \pm $ 15.4}} & 
\multicolumn{1}{|p{0.5in}}{\Centering {\fontsize{10pt}{12.0pt}\selectfont 1.3$ \pm $ 1.9}} & 
\multicolumn{1}{|p{0.5in}|}{\cellcolor[HTML]{EFEFEF}\Centering {\fontsize{10pt}{12.0pt}\selectfont \textbf{19.4}}} \\
\hhline{~---------}
%row no:6
\multicolumn{1}{|p{0.25in}}{\cellcolor[HTML]{EFEFEF}} & 
\multicolumn{1}{|p{0.4in}}{\cellcolor[HTML]{EFEFEF}\Centering {\fontsize{11pt}{13.2pt}\selectfont South}} & 
\multicolumn{1}{|p{0.47in}}{\Centering {\fontsize{10pt}{12.0pt}\selectfont -1.2$ \pm $ 0.1}} & 
\multicolumn{1}{|p{0.61in}}{\Centering {\fontsize{10pt}{12.0pt}\selectfont 21.1$ \pm $ 10.5}} & 
\multicolumn{1}{|p{0.61in}}{\Centering {\fontsize{10pt}{12.0pt}\selectfont 1.0$ \pm $ 0.6}} & 
\multicolumn{1}{|p{0.49in}}{\cellcolor[HTML]{EFEFEF}\Centering {\fontsize{10pt}{12.0pt}\selectfont \textbf{13.6}}} & 
\multicolumn{1}{|p{0.5in}}{\Centering {\fontsize{10pt}{12.0pt}\selectfont -0.2$ \pm $ 1.0}} & 
\multicolumn{1}{|p{0.5in}}{\Centering {\fontsize{10pt}{12.0pt}\selectfont -3.9$ \pm $ 2.4}} & 
\multicolumn{1}{|p{0.5in}}{\Centering {\fontsize{10pt}{12.0pt}\selectfont 0.4$ \pm $ 0.8}} & 
\multicolumn{1}{|p{0.5in}|}{\cellcolor[HTML]{EFEFEF}\Centering {\fontsize{10pt}{12.0pt}\selectfont \textbf{9.2}}} \\
\hhline{----------}
%row no:7
\multicolumn{1}{|p{0.25in}}{\multirow{1}{*}{\begin{tabular}{p{0.01in}}\Centering {\fontsize{11pt}{13.2pt}\selectfont K1}\\\end{tabular}}} & 
\multicolumn{1}{|p{0.4in}}{\Centering {\fontsize{11pt}{13.2pt}\selectfont North}} & 
\multicolumn{1}{|p{0.47in}}{\Centering {\fontsize{10pt}{12.0pt}\selectfont 0.4 $ \pm $ 1.0}} & 
\multicolumn{1}{|p{0.61in}}{\Centering {\fontsize{10pt}{12.0pt}\selectfont 8.7 $ \pm $ 1.3}} & 
\multicolumn{1}{|p{0.61in}}{\Centering {\fontsize{10pt}{12.0pt}\selectfont 0.5$ \pm $ 0.7}} & 
\multicolumn{1}{|p{0.49in}}{\Centering {\fontsize{10pt}{12.0pt}\selectfont \textbf{25.9}}} & 
\multicolumn{1}{|p{0.5in}}{\Centering {\fontsize{10pt}{12.0pt}\selectfont 0.5 $ \pm $ 0.9}} & 
\multicolumn{1}{|p{0.5in}}{\Centering {\fontsize{10pt}{12.0pt}\selectfont 14.0 $ \pm $ 2.6}} & 
\multicolumn{1}{|p{0.5in}}{\Centering {\fontsize{10pt}{12.0pt}\selectfont 0.7$ \pm $ 0.7}} & 
\multicolumn{1}{|p{0.5in}|}{\Centering {\fontsize{10pt}{12.0pt}\selectfont \textbf{28.7}}} \\
\hhline{~---------}
%row no:8
\multicolumn{1}{|p{0.25in}}{} & 
\multicolumn{1}{|p{0.4in}}{\Centering {\fontsize{11pt}{13.2pt}\selectfont South}} & 
\multicolumn{1}{|p{0.47in}}{\Centering {\fontsize{10pt}{12.0pt}\selectfont -1.1$ \pm $ 0.8}} & 
\multicolumn{1}{|p{0.61in}}{\Centering {\fontsize{10pt}{12.0pt}\selectfont 44.0$ \pm $ 2.4}} & 
\multicolumn{1}{|p{0.61in}}{\Centering {\fontsize{10pt}{12.0pt}\selectfont 1.4$ \pm $ 0.6}} & 
\multicolumn{1}{|p{0.49in}}{\Centering {\fontsize{10pt}{12.0pt}\selectfont \textbf{66.4}}} & 
\multicolumn{1}{|p{0.5in}}{\Centering {\fontsize{10pt}{12.0pt}\selectfont -1.7$ \pm $ 0.8}} & 
\multicolumn{1}{|p{0.5in}}{\Centering {\fontsize{10pt}{12.0pt}\selectfont 59.8$ \pm $ 3.2}} & 
\multicolumn{1}{|p{0.5in}}{\Centering {\fontsize{10pt}{12.0pt}\selectfont 2.2$ \pm $ 0.5}} & 
\multicolumn{1}{|p{0.5in}|}{\Centering {\fontsize{10pt}{12.0pt}\selectfont \textbf{84.4}}} \\
\hhline{----------}
%row no:9
\multicolumn{1}{|p{0.25in}}{\multirow{1}{*}{\begin{tabular}{p{0.01in}}\cellcolor[HTML]{EFEFEF}\Centering {\fontsize{11pt}{13.2pt}\selectfont O1}\\\end{tabular}}} & 
\multicolumn{1}{|p{0.4in}}{\cellcolor[HTML]{EFEFEF}\Centering {\fontsize{11pt}{13.2pt}\selectfont North}} & 
\multicolumn{1}{|p{0.47in}}{\Centering {\fontsize{10pt}{12.0pt}\selectfont -0.2$ \pm $ 0.5}} & 
\multicolumn{1}{|p{0.61in}}{\Centering {\fontsize{10pt}{12.0pt}\selectfont -15.8$ \pm $ 13}} & 
\multicolumn{1}{|p{0.61in}}{\Centering {\fontsize{10pt}{12.0pt}\selectfont 0.4$ \pm $ 0.5}} & 
\multicolumn{1}{|p{0.49in}}{\cellcolor[HTML]{EFEFEF}\Centering {\fontsize{10pt}{12.0pt}\selectfont \textbf{35.2}}} & 
\multicolumn{1}{|p{0.5in}}{\Centering {\fontsize{10pt}{12.0pt}\selectfont 0.01$ \pm $ 0.5}} & 
\multicolumn{1}{|p{0.5in}}{\Centering {\fontsize{10pt}{12.0pt}\selectfont -3.1$ \pm $ 14.1}} & 
\multicolumn{1}{|p{0.5in}}{\Centering {\fontsize{10pt}{12.0pt}\selectfont 0.1$ \pm $ 0.4}} & 
\multicolumn{1}{|p{0.5in}|}{\cellcolor[HTML]{EFEFEF}\Centering {\fontsize{10pt}{12.0pt}\selectfont \textbf{29.1}}} \\
\hhline{~---------}
%row no:10
\multicolumn{1}{|p{0.25in}}{\cellcolor[HTML]{EFEFEF}} & 
\multicolumn{1}{|p{0.4in}}{\cellcolor[HTML]{EFEFEF}\Centering {\fontsize{11pt}{13.2pt}\selectfont South}} & 
\multicolumn{1}{|p{0.47in}}{\Centering {\fontsize{10pt}{12.0pt}\selectfont 0.7$ \pm $ 0.4}} & 
\multicolumn{1}{|p{0.61in}}{\Centering {\fontsize{10pt}{12.0pt}\selectfont -28.3$ \pm $ 23}} & 
\multicolumn{1}{|p{0.61in}}{\Centering {\fontsize{10pt}{12.0pt}\selectfont 0.8$ \pm $ 0.7}} & 
\multicolumn{1}{|p{0.49in}}{\cellcolor[HTML]{EFEFEF}\Centering {\fontsize{10pt}{12.0pt}\selectfont \textbf{44.3}}} & 
\multicolumn{1}{|p{0.5in}}{\Centering {\fontsize{10pt}{12.0pt}\selectfont 1.1$ \pm $ 0.4}} & 
\multicolumn{1}{|p{0.5in}}{\Centering {\fontsize{10pt}{12.0pt}\selectfont 30.7$ \pm $ 16.5}} & 
\multicolumn{1}{|p{0.5in}}{\Centering {\fontsize{10pt}{12.0pt}\selectfont 1.0$ \pm $ 0.6}} & 
\multicolumn{1}{|p{0.5in}|}{\cellcolor[HTML]{EFEFEF}\Centering {\fontsize{10pt}{12.0pt}\selectfont \textbf{53.5}}} \\
\hhline{----------}

\end{tabular}

\caption{Model performance statistics for the main barotropic tidal components (M2, S2, K1, O1) in the forcing simulation and the state-of-the-art data-assimilated global tidal model TPX08 from OSU  in comparison to X-TRACK altimetry. dA = mean amplitude biais; dG = mean phase bias; $ \vert $ $ \vert $ dE$ \vert $ $ \vert $ \ = mean combined bias;  $ \vert $ $ \vert $ dE$ \vert $ $ \vert $ /A = mean relative bias.}
\label{Tab3_Lucie}
 \end{table}
 
%%%%%%%%%%%%%%%%%%%% Table No: 3 ends here %%%%%%%%%%%%%%%%%%%%



%%%%%%%%%%%%%%%%%%%% Table No: 4 starts here %%%%%%%%%%%%%%%%%%%%


\begin{table}[!h]
 			\centering
\begin{tabular}{p{0.69in}p{0.58in}p{0.58in}p{0.58in}p{0.42in}p{0.54in}p{0.54in}p{0.54in}p{0.5in}}
\hline
%row no:1
\multicolumn{1}{|p{0.69in}}{\Centering {\fontsize{11pt}{13.2pt}\selectfont Simulation}} & 
\multicolumn{4}{|p{2.12in}}{\Centering {\fontsize{11pt}{13.2pt}\selectfont R\textsubscript{NBQ}}} & 
\multicolumn{4}{|p{2.12in}|}{\Centering {\fontsize{11pt}{13.2pt}\selectfont Forcing (MITgcm)}} \\
\hhline{---------}
%row no:2
\multicolumn{1}{|p{0.69in}}{} & 
\multicolumn{1}{|p{0.58in}}{\Centering {\fontsize{11pt}{13.2pt}\selectfont dA (cm)}} & 
\multicolumn{1}{|p{0.58in}}{\Centering {\fontsize{11pt}{13.2pt}\selectfont dG ($ ^{\circ} $)}} & 
\multicolumn{1}{|p{0.58in}}{\Centering {\fontsize{11pt}{13.2pt}\selectfont $ \vert\vert dE \vert\vert$  (cm)}} & 
\multicolumn{1}{|p{0.42in}}{\Centering {\fontsize{11pt}{13.2pt}\selectfont \Tiny{\textbf{$ \vert\vert dE \vert\vert/A$ ($\%$)}}}} & 
\multicolumn{1}{|p{0.54in}}{\Centering {\fontsize{11pt}{13.2pt}\selectfont dA (cm)}} & 
\multicolumn{1}{|p{0.54in}}{\Centering {\fontsize{11pt}{13.2pt}\selectfont dG ($ ^{\circ} $)}} & 
\multicolumn{1}{|p{0.54in}}{\Centering {\fontsize{11pt}{13.2pt}\selectfont $ \vert\vert dE\vert\vert $ (cm)}} & 
\multicolumn{1}{|p{0.5in}|}{\Centering {\fontsize{11pt}{13.2pt}\selectfont \Tiny{\textbf{$ \vert\vert dE \vert\vert/A$ ($\%$ )}}}} \\
\hhline{---------}
%row no:3
\multicolumn{1}{|p{0.69in}}{\Centering {\fontsize{11pt}{13.2pt}\selectfont M2}} & 
\multicolumn{1}{|p{0.58in}}{\Centering {\fontsize{10pt}{12.0pt}\selectfont 3.9 $ \pm $ 0.4}} & 
\multicolumn{1}{|p{0.58in}}{\Centering {\fontsize{10pt}{12.0pt}\selectfont -1.6 $ \pm $ 1.5}} & 
\multicolumn{1}{|p{0.58in}}{\Centering {\fontsize{10pt}{12.0pt}\selectfont 2.8$ \pm $ 0.5}} & 
\multicolumn{1}{|p{0.42in}}{\Centering {\fontsize{10pt}{12.0pt}\selectfont \textbf{9.8}}} & 
\multicolumn{1}{|p{0.54in}}{\Centering {\fontsize{10pt}{12.0pt}\selectfont -4.5$ \pm $ 0.5}} & 
\multicolumn{1}{|p{0.54in}}{\Centering {\fontsize{10pt}{12.0pt}\selectfont 4.6$ \pm $ 1.0}} & 
\multicolumn{1}{|p{0.54in}}{\Centering {\fontsize{10pt}{12.0pt}\selectfont 3.6$ \pm $ 0.5}} & 
\multicolumn{1}{|p{0.5in}|}{\Centering {\fontsize{10pt}{12.0pt}\selectfont \textbf{12.7}}} \\
\hhline{---------}
%row no:4
\multicolumn{1}{|p{0.69in}}{\Centering {\fontsize{11pt}{13.2pt}\selectfont S2}} & 
\multicolumn{1}{|p{0.58in}}{\Centering {\fontsize{10pt}{12.0pt}\selectfont 0.6$ \pm $ 0.3}} & 
\multicolumn{1}{|p{0.58in}}{\Centering {\fontsize{10pt}{12.0pt}\selectfont -1.5$ \pm $ 0.9}} & 
\multicolumn{1}{|p{0.58in}}{\Centering {\fontsize{10pt}{12.0pt}\selectfont 0.5$ \pm $ 0.2}} & 
\multicolumn{1}{|p{0.42in}}{\Centering {\fontsize{10pt}{12.0pt}\selectfont \textbf{4.7}}} & 
\multicolumn{1}{|p{0.54in}}{\Centering {\fontsize{10pt}{12.0pt}\selectfont 1.1$ \pm $ 0.3}} & 
\multicolumn{1}{|p{0.54in}}{\Centering {\fontsize{10pt}{12.0pt}\selectfont 2.8$ \pm $ 0.6}} & 
\multicolumn{1}{|p{0.54in}}{\Centering {\fontsize{10pt}{12.0pt}\selectfont 0.9$ \pm $ 0.2}} & 
\multicolumn{1}{|p{0.5in}|}{\Centering {\fontsize{10pt}{12.0pt}\selectfont \textbf{8.4}}} \\
\hhline{---------}
%row no:5
\multicolumn{1}{|p{0.69in}}{\Centering {\fontsize{11pt}{13.2pt}\selectfont K1}} & 
\multicolumn{1}{|p{0.58in}}{\Centering {\fontsize{10pt}{12.0pt}\selectfont -1.3$ \pm $ 0.6}} & 
\multicolumn{1}{|p{0.58in}}{\Centering {\fontsize{10pt}{12.0pt}\selectfont -5.1$ \pm $ 0.1}} & 
\multicolumn{1}{|p{0.58in}}{\Centering {\fontsize{10pt}{12.0pt}\selectfont 0.9$ \pm $ 0.4}} & 
\multicolumn{1}{|p{0.42in}}{\Centering {\fontsize{10pt}{12.0pt}\selectfont \textbf{24.2}}} & 
\multicolumn{1}{|p{0.54in}}{\Centering {\fontsize{10pt}{12.0pt}\selectfont -1.0$ \pm $ 0.3}} & 
\multicolumn{1}{|p{0.54in}}{\Centering {\fontsize{10pt}{12.0pt}\selectfont 0.2$ \pm $ 0.0}} & 
\multicolumn{1}{|p{0.54in}}{\Centering {\fontsize{10pt}{12.0pt}\selectfont 0.7$ \pm $ 0.2}} & 
\multicolumn{1}{|p{0.5in}|}{\Centering {\fontsize{10pt}{12.0pt}\selectfont \textbf{20.6}}} \\
\hhline{---------}
%row no:6
\multicolumn{1}{|p{0.69in}}{\Centering {\fontsize{11pt}{13.2pt}\selectfont O1}} & 
\multicolumn{1}{|p{0.58in}}{\Centering {\fontsize{10pt}{12.0pt}\selectfont -1.2$ \pm $ 0.1}} & 
\multicolumn{1}{|p{0.58in}}{\Centering {\fontsize{10pt}{12.0pt}\selectfont 21.1$ \pm $ 10.5}} & 
\multicolumn{1}{|p{0.58in}}{\Centering {\fontsize{10pt}{12.0pt}\selectfont 1.0$ \pm $ 0.6}} & 
\multicolumn{1}{|p{0.42in}}{\Centering {\fontsize{10pt}{12.0pt}\selectfont \textbf{47.5}}} & 
\multicolumn{1}{|p{0.54in}}{\Centering {\fontsize{10pt}{12.0pt}\selectfont -0.7$ \pm $ 0.6}} & 
\multicolumn{1}{|p{0.54in}}{\Centering {\fontsize{10pt}{12.0pt}\selectfont 16.6$ \pm $ 8.1}} & 
\multicolumn{1}{|p{0.54in}}{\Centering {\fontsize{10pt}{12.0pt}\selectfont 0.6$ \pm $ 0.4}} & 
\multicolumn{1}{|p{0.5in}|}{\Centering {\fontsize{10pt}{12.0pt}\selectfont \textbf{33.5}}} \\
\hhline{---------}

\end{tabular}

%%%%%%%%%%%%%%%%%%%% Table No: 4 ends here %%%%%%%%%%%%%%%%%%%%

\caption{Model performance statistics for the main barotropic tidal components (M2, S2, K1, O1) in \textbf{R\textsubscript{NBQ}} and the forcing simulation in comparison to tidal gauges. dA = mean amplitude biais; dG = mean phase bias; $ \vert $ $ \vert $ dE$ \vert $ $ \vert $ \ = mean combined bias;  $ \vert $ $ \vert $ dE$ \vert $ $ \vert $ /A = mean relative bias.}
\label{Tab4_Lucie}

 \end{table}

\vspace{\baselineskip}
\subsection{Internal fine-scale processes }

\vspace{\baselineskip}
We now focus on internal fine-scale processes generated by tidal flows. In Gibraltar-strait area, such processes are subject to a strong neap-spring tidal variability. Hence, to study their dynamic, three regimes of tidal forcing during the modelled time period (Figure \ref{Fig3_Lucie}) can be distinguished:\par

\begin{itemize}
	\item a week tidal forcing when the minimal barotropic\ velocity above CS crest  during the tidal outflow is superior to -0.5 m/s (green curve)\par

	\item a moderate tidal forcing when the minimal barotropic\ velocity above CS crest  during the tidal outflow is inferior to -0.5 m/s and superior to -1.5 m/s (blue curve)\par

	\item a strong tidal forcing when the minimal barotropic\ velocity above CS crest  during the tidal outflow is inferior to -1.5 m/s (red curve)
\end{itemize}
During the simulated time-period (1 month), the moderate tidal forcing is the dominant regime.\\
\vspace{\baselineskip}


%%%%%%%%%%%%%%%%%%%% Figure/Image No: 2 starts here %%%%%%%%%%%%%%%%%%%%

\begin{figure}[!h]
	\begin{Center}
		\includegraphics[width=6.3in,height=2.75in]{./media/image2.jpg}
	\end{Center}

%%%%%%%%%%%%%%%%%%%% Figure/Image No: 2 Ends here %%%%%%%%%%%%%%%%%%%%

\caption{Barotropic velocity above CS crest. The colored curves differentiate the three different tidal regimes : green curves - weak tidal forcing, blue curves - moderate tidal forcing and red curves - strong tidal forcing. The circle markers indicate the acquisition date of the high-resolution satellite images on\  figures \ref{Fig4_Lucie} and \ref{Fig5_Lucie}.}
\label{Fig2_Lucie}
\end{figure}

\subsubsection{Comparison to high-resolution satellite images}

\vspace{\baselineskip}
An effective procedure to trace the evolution of the internal or baroclinic field consists of monitoring the spatial gradient of surface velocity. The superposition of baroclinic and barotropic currents gives rise to areas of strong horizontal convergence/divergence of the flow characterized by short-scale surface roughness which can be captured by SARs [Alpers, 1985] and MSI- Multispectral Imager. This allows the identification of baroclinic structures such as internal hydraulic jumps or internal propagating internal waves through the observation of the ocean surface. \par


\vspace{\baselineskip}
Multiple SAR and RGB images have been acquired from Sentinel-1 and Sentinel-2 satellites during the modelled time period (10/09/2017 - 10/10/2017). We focus on two particular satellite images acquired both in moderate regime but at different phases of the tidal cycle: \par

- a SAR image acquired on the 18/09/2017 – 06:27:42\textit{\  }at the end of the tidal cycle\par

- a RGB image acquired \textit{ }the 19/09/2017 – 11:18:00 just after the tidal maximal outflow. \par


\vspace{\baselineskip}
The SAR image was taken at the end of a moderate tidal cycle when the well-known train of ISWs propagating eastward reaches the bay of Gibraltar (Figure \ref{Fig4_Lucie} - Left panel). The train of ISWs on the SAR image is composed at least of 4 distinct solitary waves. On this particular SAR image, a second feature is identifiable as a westward propagating internal wave in Tangier Bassin. The origin of these westward propagating internal waves will be discuss in section 2.4.3.\par

\vspace{\baselineskip}
In R\textsubscript{NBQ}, the train is composed solely of one or two solitary waves (figure \ref{Fig4_Lucie} - Right) however it reaches the bay of Gilbratar at approximately the same moment. Hence, even if the number of solitary waves in the train is underestimated in R\textsubscript{NBQ}, the propagation speed of the train seems to be well represented. Moreover, the westward propagating internal wave observed in Tangier Bassin are also quite well represented in R\textsubscript{NBQ} (location, orientation, shape). The resolution of the SAR image is 10 m, compared with a resolution of 220 m for R\textsubscript{NBQ}. The resolution of R\textsubscript{NBQ} seems therefore insufficient to represent explicitly the smallest solitary waves of the train. 
\par

%%%%%%%%%%%%%%%%%%%% Figure/Image No: 3 starts here %%%%%%%%%%%%%%%%%%%%

\begin{figure}[!h]
	\begin{Center}
		\includegraphics[width=6.3in,height=3.43in]{./media/image3.png}
	\end{Center}

%%%%%%%%%%%%%%%%%%%% Figure/Image No: 3 Ends here %%%%%%%%%%%%%%%%%%%%

\caption{Left – Sentinel-1 Synthetic Aperture Radar (SAR) image acquired on the 18/09/2017 – 06:27:42 and processed by ESA. Right - Divergence of the surface velocity in R\textsubscript{NBQ} the 18/09/2017 – 06:25:00 after a moderate tidal outflow. }
\label{Fig3_Lucie}
\end{figure}

\vspace{\baselineskip}
The RGB image (Figure \ref{Fig5_Lucie} - Left panel) was taken just after the maximum of a moderate tidal outflow when hydraulic jumps are presumably formed in the lee-side of the main sill. On this particular RGB image, instead of the expected single wavefront, two distinct fronts can be observed in CS area : one downstream of CS extending all across the strait and another one located upstream of CS and of smaller extension. At this stage of the tidal cycle, the train of ISWs generated during the previous tidal outflow is still propagating eastward inside the Alboran sea and is made of a larger number of solitary waves than in the above SAR image.\par

\vspace{\baselineskip}
In R\textsubscript{NBQ} (Figure \ref{Fig5_Lucie} - Right), the train of ISWs propagating in the Alboran sea is composed of three solitary waves (still less than on the RGB image). Its propagation speed seems to be well represented, whereas the front deformation is slightly different. 
In R\textsubscript{NBQ}, the two distinct wave fronts observed above CS on the RGB image are located a little further west. At this time in R\textsubscript{NBQ}, the tidal current above CS turned sub-critical and the hydraulic jumps formed above CS have already been released leading to an eastward propagation of the two wave fronts. In so far as some barotropic components (such as N2) are not taken into account in R\textsubscript{NBQ}, it seems that the tidal current above CS is slightly underestimated in R\textsubscript{NBQ}.

\vspace{\baselineskip}


%%%%%%%%%%%%%%%%%%%% Figure/Image No: 4 starts here %%%%%%%%%%%%%%%%%%%%

\begin{figure}[!h]
	\begin{Center}
		\includegraphics[width=6.3in,height=3.22in]{./media/image4.png}
	\end{Center}

%%%%%%%%%%%%%%%%%%%% Figure/Image No: 4 Ends here %%%%%%%%%%%%%%%%%%%%

\caption{Left - BOA reflectance image (RGB composites) on 19/09/2017 11:18:08, re-processed from Level-2A product for Sentinel-2A/MSI instrument distributed by ESA – Right -  Divergence of the surface velocity in R\textsubscript{NBQ}, the 19/09/2017 – 11:18:00 at a moderate tidal outflow. }
\label{Fig4_Lucie}
\end{figure}

\vspace{\baselineskip}
\subsubsection{Dynamics of the hydraulic jumps}

%%%%%%%%%%%%%%%%%%%% Figure/Image No: 5 starts here %%%%%%%%%%%%%%%%%%%%

\begin{figure}[!h]
	\begin{Center}
		\includegraphics[width=6.61in,height=4.99in]{./media/image5.jpg}
	\end{Center}

%%%%%%%%%%%%%%%%%%%% Figure/Image No: 5 Ends here %%%%%%%%%%%%%%%%%%%%


\caption{ Lower pannels\ -  Divergence of the flow in \textbf{R\textsubscript{NBQ}} at the maximum of a weak tidal outflow (c), a moderate tidal outflow (f) and a strong tidal outflow (d). The black line locates the vertical section 1 (S1) and section 2 (S2). Middle pannels - Section 2 (S2) of the horizontal velocity field in \textbf{R\textsubscript{NBQ}} at the maximum of a weak tidal outflow (b), a moderate tidal outflow (e) and a strong tidal outflow (h). Upper pannels - Section 1 (S1) of the horizontal velocity field in \textbf{R\textsubscript{NBQ}} at the maximum of a weak tidal outflow (a), a moderate tidal outflow (d) and a strong tidal outflow (g). Red contour represent super-critical flow area (F1=c1).}
\label{Fig5_Lucie}
\end{figure}

\vspace{\baselineskip}
Under weak tidal forcing, only a small part of the bottom layer is super-critical above CS crest (partial hydraulic control figure \ref{Fig5_Lucie}.a-b). The surface layer remains sub-critical and there is no signature of any hydraulic jumps at the ocean surface (figure \ref{Fig5_Lucie}.c). \par

\vspace{\baselineskip}
Under moderate tidal forcing (figure \ref{Fig5_Lucie}.d-e), the whole water column is super-critical above CS (total hydraulic control) leading to the formation of large hydraulic jumps. The signature of these hydraulic jumps is visible on the divergence of the surface flow (figure \ref{Fig5_Lucie}.f). We can indeed observe a strong asymmetry between the northern and southern parts of the strait (due to geographical constriction) :\par

\begin{itemize}
	\item in the northern part (figure \ref{Fig5_Lucie}.d), we observe a single super-critical area above CS,\par

	\item in the southern part (figure \ref{Fig5_Lucie}.c), we observe two distinct super-critical regions above CS.
\end{itemize}\par

Hence a moderate tidal outflow is characterized by the formation of two hydraulic jumps: one located downstream of CS extending all across the strait and one located upstream of CS and confined to the southern half section of the strait\par

\vspace{\baselineskip}
Under strong tidal forcing (spring tides), when the outflow is maximal, the southern upstream hydraulic jump (located initially around\  -5.73$ ^{\circ} $ \ of longitude) is swept down to the lee side of the sill (around  -5.76$ ^{\circ} $  of longitude), and in the southern part of the strait, the flow acquires the hydraulic configuration of a pure approach-controlled state, with only one supercritical region extending from the upstream control section to a few hundred meters downstream (of CS).\par

\vspace{\baselineskip}
\subsubsection{ ISWs formation: a strong neap-spring tide variability}

\vspace{\baselineskip}
 During moderate tides (figure \ref{Fig7_Lucie}.b-e), in the middle part of the strait (section 1), an internal hydraulic jump is formed on the downstream side of CS at maximum tidal ouflow. When the tidal flow reverses, the hydraulic jump is released and leads to the eastward propagation of an internal bore (IB). It is then subject to multiple reflections (rW) in Tarrifa Narrows and degenerates into solitary waves (ISWs).\par

\vspace{\baselineskip}
However, during weak neap-tides (figure \ref{Fig7_Lucie}.a-d); the favorable hydraulic conditions for the generation of lee waves (LWs) in the lee-side of CS inhibit the internal bore generation.  \par


\vspace{\baselineskip}
During spring tides (figure \ref{Fig7_Lucie}.c-f), in the middle part of the strait, two internal hydraulic jumps are formed above CS: one upstream (uHJ) and another downstream (dHJ) of the sill. When the tidal flow reverses, the hydraulic jumps are released and lead to the eastward propagation of two internal bores (IB). The largest internal bore (emitted on the downstream side) propagates faster and catches up the smallest one (emitted on the upstream side). It is then subject to multiple reflections (rW) in Tarrifa Narrows and degenerates into solitary waves (ISWs).\par

\vspace{\baselineskip}

\vspace{\baselineskip}

\vspace{\baselineskip}


%%%%%%%%%%%%%%%%%%%% Figure/Image No: 6 starts here %%%%%%%%%%%%%%%%%%%%

\begin{figure}[!h]
	\begin{Center}
		\includegraphics[width=6.3in,height=4.32in]{./media/image6.jpg}
	\end{Center}

%%%%%%%%%%%%%%%%%%%% Figure/Image No: 6 Ends here %%%%%%%%%%%%%%%%%%%%

\caption{ Upper panels \textbf{- }Section 1 of the density field in \textbf{R\textsubscript{NBQ}} after a weak tidal outflow (a), a moderate tidal outflow (b) and a strong tidal outflow (c). Lower panels - Space-time diagram of the vertical isopycnal displacement for $ \rho $ =1031.1 kg/m3 (bold black line on the upper panels) along the section 1 during two weak tidal cycle (a), a moderate and a weak tidal cycle (b) and two strong tidal cycle (c). The white arrows represent the mean tidal inflow above Camarinal sill whereas colored arrows represent the mean tidal outflow above Camarinal sill (green for weak, blue for moderate and red for strong). Horizontal black lines locate temporally the vertical section of the density field (upper panels) whereas colored dotted\ lines locate temporally the maximum of the previous tidal outflow (section 1 on Fig.  \ref{Fig5_Lucie}). }
\label{Fig7_Lucie}
\end{figure}

%\vspace{\baselineskip}

%\printbibliography
%\end{document}

 
\selectlanguage{french}


%%%%%%%%%%%%%%%%%%%%%%%%%%%%%%%%%%%%%%%%%%%%%%%%%%%%%%%%%%%%%%%%%%%
%%%%%%%%%%%%%%%%%%%%%%%%%%%%%%%%%%%%%%%%%%%%%%%%%%%%%%%%%%%%%%%%%%%
%%% Maquette 3D NH-HR
%%%%%%%%%%%%%%%%%%%%%%%%%%%%%%%%%%%%%%%%%%%%%%%%%%%%%%%%%%%%%%%%%%%
%%%%%%%%%%%%%%%%%%%%%%%%%%%%%%%%%%%%%%%%%%%%%%%%%%%%%%%%%%%%%%%%%%%
\newpage
\null
\newpage
\section{Maquette 3D NH-HR \textit{(Tâche 1.2)}}
\label{section3DNHNR}

\noindent\fbox{\noindent\begin{minipage}{1\textwidth}
\noindent La maquette tri-dimensionnelle NH-HR est la maquette la mieux résolue sur l'horizontale (45 m) et la verticale (40 niveaux $\sigma$) implémentée dans le cadre du présent contrat.  De façon originale, cette maquette permet la simulation des "grandes structures turbulentes" \textit{(LES)}. Elle est pour cela basée sur le coeur non-hydrostatique et non-Boussinesq CROCO et met en oeuvre les schémas d'advection d'ordre supérieur et les schémas de diffusion turbulente isotropes disponibles dans CROCO, trois éléments essentiels et indispensables pour simuler numériquement à minima les instabilités primaires amorçant la cascade turbulente directe. Une comparaison détaillée de la dynamique à haute résolution avec celle simulée avec la maquette à plus basse résolution (NH-REF, section \ref{section3DNHREF}) montre que les processus de grande échelle sont représentés de façon comparable dans les deux maquettes. La simulation NH-HR permet en outre un raffinement du train d'ondes solitaires et surtout une représentation cohérente et explicite des instabilités de Kelvin-Helmholtz générées au voisinage du jet méditerranéen. 
\end{minipage}
}

\subsection{Caractéristiques de la maquette NH-HR}

Les caractéristiques numériques de la maquette à plus haute résolution NH-HR sont présentées dans le tableau \ref{tab_XEC}. Cette maquette est environ 4 fois mieux résolue que la maquette à basse résolution NH-REF dans chacune des deux directions horizontales (45 m contre 220 m) et la résolution verticale reste la même  (40 niveaux verticaux de type $\sigma$). Les épaisseurs minimales et maximales des couches $\sigma$ sont fournies dans cette table: la résolution verticale est la maquette est de l'ordre de la dizaine de mètre dans la région du détroit, et elle 2 fois plus grande que la résolution horizontale dans les zones les plus profondes.\\

\begin{table}[!h]
        %\begin{minipage}{.6\textwidth}
        \centering
        \begin{tabular}{|p{\linewidth/3}|c|c|}
                \hline
                Coeur numérique & \multicolumn{2}{c|} {CROCO-NBQ} \\
                Domaine & \multicolumn{2}{c|} {6°4.8'W  5°3.4'W ;}\\
                & \multicolumn{2}{c|} {35°23.8'N  36°27.4'N}\\
                Nombre de points de grille horizontaux & \multicolumn{2}{c|} {2049x2621}  \\
                Nombre de niveaux verticaux ($\sigma$) & \multicolumn{2}{c|} {40} \\
                %\hline
                $\Delta x = \Delta y$ & \multicolumn{2}{c|} {45 m}\\
                %\hline
                Depth & Min & Max\\
                %\cline{2-3}
                & 26 m & 960 m\\
                %   \hline
                $\Delta$z & 0.7 m & 24 m\\
                Nombre de coeurs & \multicolumn{2}{c|} {447 (+1 serveur xios)}\\
                Pas de temps interne ($\Delta t_s$) & \multicolumn{2}{c|} {1 s}\\
                Pas de temps externe ($\Delta t_f$) & \multicolumn{2}{c|} {1/11 s}\\
                Schémas d'advection & \multicolumn{2}{c|} {WENO-5 (T-S) , TVD (U-V-W)} \\
                Coefficient de viscosité verticale $\nu$ & \multicolumn{2}{c|} {10$^{-6}$ m^{2}/s} \\
                Coefficient de mélange vertical $K_\rho$ & \multicolumn{2}{c|} {10$^{-6}$ m^2/s}\\
                $C_s$ & \multicolumn{2}{c|} {400 m/s}\\
                Ondes de marée & \multicolumn{2}{c|} { $\text{M}_{\text{2}}$, $\text{S}_{\text{2}}$,                            $\text{K}_{\text{1}}$, $\text{O}_{\text{1}}$ }\\
                \hline
        \end{tabular}
        \captionof{table}{Principales caractéristiques de la maquette NH-HR.}
        \label{tab_NH-HR}
        %\end{minipage}
\end{table}

L'hydrologie, les courants de marée et la circulation générale sont forcés comme pour la maquette NH-REF à partir d'une simulation hydrostatique "Mère" MitGcm du modèle opérationnel de la Méditerranée et de la Mer Noir de l'ENEA, gracieusement fournie par l'équipe de G. Sannino (ENEA, Rome)\footnote{http://www.enea.it/it/seguici/pubblicazioni/pdf-volumi/cresco-report-2016.pdf}. Ce forçage est utilisé comme condition initiale et comme conditions aux frontières ouvertes. Cette simulation "Mère" a une résolution horizontale variable d'environ 700 m dans la région du détroit, soit environ une quinzaine de fois supérieure à celle de la maquette à haute résolution NH-HR. Ce saut en résolution relativement élevé est associé à un changement de coordonnées verticales (le MitGcm est basé sur une grille à niveaux z, CROCO sur une grille $\sigma$).  Les cœurs dynamiques utilisés dans les simulations "Mère" et NH-HR sont de plus différents: la maquette NH-HR est en effet non-hydrostatique et non-Boussinesq alors que la version du modèle MitGcm utilisée pour réaliser la simulation "Mère" est hydrostatique et Boussinesq.\\
L'ensemble de ces différences est à l'origine d'un choc violent lorsque la simulation NH-HR est initialisée et forcée directement à partir de la simulation Mère MitGcm. Un tel choc entraîne une période de \textit{spin-up} de plusieurs heures durant lesquelles la qualité des caractéristiques des masses d'eaux méditerranéenne et atlantique est peut-être fortement dégradée. En conséquence, un protocole de simulation spécifique a été mis en place:
\begin{itemize}
	\item La simulation à haute résolution (HR) est initialisée et forcée par la simulation "Mère" MitGcm.
	\item Durant les 6 premières heures de simulation, le coeur hydrostatique, Boussinesq de CROCO est implémenté, limitant ainsi le choc initial et permettant une première "mise en équilibre" des champs initiaux sur la bathymétrie à haute résolution.
	\item La simulation HR est alors redémarrée (\textit{Hot Restart}) en version non-Hydrostatique et non-Boussinesq.
\end{itemize}
Quatre ondes de marées sont forcées aux frontières latérales: ($\text{M}_{\text{2}}$, $\text{S}_{\text{2}}$, $\text{K}_{\text{1}}$, $\text{O}_{\text{1}}$). Ce forçage est intégré dans les sorties issues de la simulation MitGcm à 700 m.\\

\subsection{Évaluation des coûts de calcul}

 La résolution et l'extension horizontale de la maquette NH-HR sont choisies de telle sorte que les structures dynamiques les plus fines possibles puissent être explicitement simulées dans la région du détroit de Gibraltar pour un coût de calcul "raisonnable" avec le coeur non-hydrostatique et non-Boussinesq de CROCO. Le coût relatif de calcul de la maquette NH-HR est d'environ 0.8 (0.8 heures de calcul pour 1 heure simulée sur 447 cœurs sur la machine Datarmor). Le sur-coût associé au coeur non-hydrostatique et non-Boussinesq est d'environ 3 par rapport au coeur hydrostatique CROCO.\\
Ce ratio de 0.8 autorise la simulation de plusieurs périodes de marée semi-diurne et permet par conséquent l'étude détaillée du contrôle hydraulique, de la génération d'instabilités primaires et d'ondes de gravité internes de grande amplitude dans la région du détroit. Une analyse harmonique plus fine des principales composantes de marée nécessiterait toutefois une simulation de plus longue durée.

\subsection{Évaluation de la qualité des principales ondes de marée simulées}
La qualité des ondes de marée simulées dans la simulation NH-HR est évaluée en deux temps:
\begin{itemize}
	\item Une inter-comparaison des anomalies de surface simulées dans les maquettes NH-REF et NH-HR a été réalisée sur l'ensemble du domaine numérique. L'erreur quadratique moyenne est d'environ 5 cm entre les deux maquettes, avec des valeurs plus élevées à l'est et au nord. 
	\item Deux marégraphes disponibles dans la région simulée sont utilisés pour évaluer les erreurs l'élévation du niveau de la mer. Les résultats de cette évaluation sont détaillés au chapitre \ref{chapitredelivrables}, section \ref{mareedelivrable} du présent rapport.
\end{itemize}

\subsection{Ressaut hydraulique}

La maquette NH-HR simule correctement le contrôle hydraulique localisé dans la région du seuil de Camarinal. Un tel contrôle est attendu dans cette région lorsque les courants de marée sont suffisamment puissants pour que le flot au-dessus du seuil devienne supercritique. Ce processus intervient à chaque période de marée durant 3 à 4 heures lorsque les courants de marée sont orientés vers l'Atlantique Nord (à l'exception notable de certaines périodes en régime de mortes-eaux durant lesquelles ces courants sont trop faibles). Le contrôle entraîne la formation d'un ressaut hydraulique dont la géométrie va dépendre de l'intensité des courants de marée. \\
La figure \ref{fig_ressaut_NH-HR}.a présente une coupe verticale dans la région du Seuil de Camarinal lors d'une période d'\textit{outflow} maximum simulée avec la maquette NH-HR, et la figure \ref{fig_ressaut_NH-HR}.b la trace visible du ressaut en terme d'élévation de la surface libre dans la simulation. Le ressaut est visible sur la coupe vers -5.755° longitude dans la région où les surfaces isopycnales présentent une discontinuité d'une amplitude d'environ 70 m au-dessus du seuil.\\

\subsection{Instabilités primaires (\textit{LES})}

Dans la zone du jet méditerranéen sur le flanc ouest du seuil de Camarinal, la maquette NH-HR montre l'apparition d'instabilités primaires de type Kelvin-Helmholtz, telles qu'elles ont pu être observées par Wesson and Gregg (1994). Ces instabilités n'étaient pas présentes dans la simulation plus basse résolution (220 m) réalisée à partir de la maquette NH-REF. La présence potentielle de telles instabilités est confirmée par le calcul du nombre de Richardson, inférieur à 1/4 dans la région du ressaut. La faiblesse du nombre de Richardson s'explique par le fort cisaillement entre le jet Méditerranéen (vitesse supérieure à 2 m/s) et l'écoulement des eaux Atlantiques (de l'ordre de quelques dizaines de centimètres par seconde).\\
La figure \ref{Coupe_Melange} prolonge la coupe de la figure \ref{fig_ressaut_NH-HR} dans le Bassin de Tanger et au-dessus du Seuil d'Espartel, et montre une comparaison du champs de masse volumique dans la maquette à haute résolution NH-HR et dans la maquette NH-REF.\\
Une différence importante est l'augmentation notable de l'épaisseur du jet méditerranéen dans la simulation à haute résolution en aval (pour le jet méditerranéen) de chacun des accidents bathymétriques présents dans cette section verticale. La remontée des surfaces isopycnales observée sur cette figure pour la simulation à haute résolution peut sembler paradoxale puisque, pour des contraintes bathymétriques quasi-équivalentes, une augmentation de la seule résolution horizontale devrait plutôt conduire à une représentation plus fine du jet méditerranéen. Une explication pourrait être apportée par la présence de nombreuses instabilités de type Kelvin-Helmholtz se développant dans cette région. Ces instabilités primaires amorcent la cascade turbulente directe et entraînent un brassage \textit{(stirring)} explicitement simulé avec une résolution de 45 m. Ce brassage est finalement lui-même à l'origine d'une augmentation du mélange. Dans la simulation à plus basse résolution, ces instabilités constituent des processus "sous-maille", mal résolus pour lesquels un schéma de turbulence plus adapté devrait être mis en place. Ces résultats très préliminaires doivent maintenant être confirmés et de nouveaux diagnostiques sont en cours d'implémentation afin, en particulier, de mieux comprendre l'impact de la simulation des grandes structures turbulentes sur le mélange.\\


\begin{figure}[!h]
        \includegraphics[width=\textwidth]{./ressaut_2D_it790_vhr_IE2.png}
        \caption{a) Coupe verticale du seuil de Camarinal. Couleur : vorticité d'axe y, traits plein : isopycnes, et vecteurs vitesse u-w. b) Élévation de la surface libre simulée avec la maquette NH-HR}
        \label{fig_ressaut_NH-HR}
        \end{figure}

\begin{figure}[!h]
        \includegraphics[width=\textwidth]{./media/comp_GExO_hr-vhr_IE.png}
        \caption{Coupe verticale selon la section S2 de la figure \ref{fig_mod1_HR-REF}.b de la masse volumique dans les 2 maquettes 3D en $kg.m^{-3}$. En couleur: simulation NH-HR (résolution dec45 m, 40 niveaux $\sigma$), contours pointillés: simulation NH-REF (résolution de 220 m, même nombre de niveaux $\sigma$).}
        \label{Coupe_Melange}
        \end{figure}
  	
\subsection{Ondes internes de grande amplitude}

Comme son homologue à plus basse résolution NH-REF, la maquette à haute résolution NH-HR est permet de simuler la génération et la propagation des modes 1 et 2 d'ondes internes de gravité dans le détroit de Gibraltar, ce qui est en accord avec les observations des satellites Sentinel 1 et 2. De façon similaires, des ondes de grande amplitude (de l'ordre de la centaine de mètres), sont générées lors de la relaxation du ressaut hydraulique au Seuil de Camarinal. Ces ondes ont une signature de mode 1. Elles sont précédées par la propagation d'un mascaret interne puis se décomposent en train d'ondes solitaires sous l'action de la dispersion non-hydrostatique qui vient équilibrer les non-linéarités liées à l'advection. Au cours de sa propagation dans le détroit de Tarifa, ce train d'onde va en partie se réfléchir sur les côtes en train ondes se propageant vers l'ouest. Afin de détailler l'apport de l'augmentation de la résolution sur ces ondes, la figure \ref{fig_mod1_HR-REF}.a présente une coupe verticale en densité du train d'ondes solitaires alors qu'il entre en Mer d'Alboran dans la maquette NH-HR (en couleur) et dans la maquette NH-REF (en pointillés). \\
Les ondes de mode 1 se sont propagées plus vite dans la maquette NH-HR, et le nombre d'ondes dans le train est plus grand (5 contre 2). Ce comportement est le même à chaque période de marée et la variation du nombre de solitons d'une période de marée à une autre peut être simulée avec la maquette NH-HR. Cette forte variabilité diurne est illustrée dans le tableau \ref{tab_XEC}, réalisé en notant l'arrivée de l'onde de mode 1 au point le plus à l'est dans la figure \ref{fig_mod1_HR-REF}.a (ou à tous les points de la figure \ref{fig_mod1_HR-REF}.b pour le calcul de la vitesse de propagation). Les vitesses les plus fortes dans ce tableau correspondent à une advection par des courants de marée plus importants, on observe que ces périodes sont associées à un plus petit nombre d'ondes dans le train avec une périodicité elle aussi plus petite. Il est à noter que ces valeurs ne sont pas identiques pour toutes les latitudes à la sortie du détroit: du fait de la dispersion induite par la force de Coriolis, l'amplitude de la première onde est par exemple plus grande dans la partie sud.

\begin{figure}[!h]
 	\includegraphics[width=\textwidth]{media/comp_train_it36_IE2_hr-vhr.png}
 	\caption{a) Coupe verticale selon la section S1 de densité (kg/m$^3$) dans les maquettes NH-REF (pointillés) et NH-HR (couleurs). b) Surface libre simulée avec la maquette NH-HR avec trace des sections S1 et S2 et points ayant servis pour la table \ref{tab_XEC}.}
 	\label{fig_mod1_HR-REF}
\end{figure}

\begin{table}[!h]
	%\begin{minipage}{.6\textwidth}
	\centering
	\begin{tabular}{|l|c|c|c|c|}
		\hline
		Numéro du train &  1 & 2 & 3 & 4  \\
		\hline
		Temps écoulé depuis le passage du train précédent (jour) & & 0.42 &0.6 & 0.41\\
		Vitesse dans le détroit de Tarifa (m/s) & 1.6 & 1.9 & 1.6 & 2.1\\
		Nombre d'ondes constituant le train & 11 & 5 & 8 & 2 \\
		Amplitude du premier soliton (m) & 50 &70 &45 &40\\
		Période moyenne (mn) & 20 & 12 & 18 & 12.5\\
		\hline
	\end{tabular}
	\captionof{table}{Caractéristiques train d'ISW sortie est du détroit pour conditions MM}
	\label{tab_XEC}
	%\end{minipage}
\end{table}

\subsection{Génération de tourbillons dans le sillage des solitons}

Les simulations NH-REF et NH-HR montrent la génération d'un tourbillon au passage des trains de solitons à la sortie du détroit. Cette génération pourrait être une conséquence des violents cisaillements de courants horizontaux générés par l'onde interne.\\
La structure tourbillonnaire demeure dans cette région puis déforme le train d'ondes solitaires suivant. A notre connaissance, il n'existe pas d'observations ni de preuves directes de son existence.\\
Son impact sur les solitons n'est toutefois pas négligeable puisque les simulations numériques montrent pour certaines périodes de marées une désorganisation du train d'ondes solitaires à la sortie du détroit. Cette déstructuration du train d'ondes solitaires pourrait être une conséquence de son interaction avec le tourbillon suivant ou être associée à la génération de la structure tourbillonnaire. Ces conséquences ont été observées par (Vlasenko et al., 2009) et pourraient quoiqu'il en soit confirmer indirectement la présence de la structure tourbillonnaire. Il convient toutefois d'être prudent et de nouvelles investigations sont actuellement en cours afin de préciser le mécanisme de génération du tourbillon et de rassembler des observations in situ pouvant confirmer sa présence.

\subsection{Sensibilité au protocole de simulation et évolution de la maquette}
La maquette NH-HR n'est bien sûr pas figée et plusieurs évolutions sont actuellement en cours:
\begin{itemize}
 \item La bathymétrie utilisée dans la maquette NH-HR du présent rapport est celle de la grille HOMONIM résolue à 500 m. Une bathymétrie résolue à 100 m du détroit a plus récemment été fournie et est en cours d'implémentation et de test dans la simulation NH-HR.
 \item La maquette NH-HR a été utilisée pour simuler la dynamique du détroit dans différents régimes de marées (mortes-eaux, vives-eaux et marée moyenne), présentés au chapitre \ref{chapitredelivrables}. Leur exploitation est actuellement en cours.
 \item L'impact de la stratification et de son évolution saisonnière sera explorée prochainement en choisissant une initialisation à une date différente dans les champs fournis par l'ENEA. Cette étude sera toutefois associée au retour du forçage atmosphérique afin de simuler de la façon la plus réaliste possible les mécanismes locaux de re-stratification. Comme indiqué en introduction du présent rapport, le groupe CROCO-Aérologie a en effet souhaité aborder par étapes l'étude de la dynamique dans la région du détroit. Lorsque cela a un sens, les forçages et par conséquent les processus et autres mécanismes physiques sont introduits progressivement. L'objectif d'une telle approche est double: mieux comprendre la dynamique locale et mieux contrôler le protocole de modélisation numérique.
\end{itemize}

\subsection{Discussion, conclusions}

La maquette NH-HR permet, de façon attendue, de raffiner la simulation des ondes solitaires se propageant en direction de la Méditerranée depuis le détroit de Gibraltar. Une étude plus détaillée de ces ondes de grande amplitude à partir d'observations dédiées et, ou, d'une modélisation simultanée avec un modèle simplifié de type KdV devrait fournir de précieuses informations sur l'impact de la résolution sur la dissipation physique et la dissipation numérique, sur la dispersion non-hydrostatique ou encore sur la bonne représentation des processus non-linéaires.\\
La maquette  NH-HR autorise en outre de façon originale la simulation explicite des "grandes structures turbulentes" (simulation dite \textit{LES}) dans la région du détroit de Gibraltar. L'augmentation de la résolution entre les maquettes NH-REF et NH-HR autorise en effet la simulation des structures tourbillonnaires associées aux instabilités primaires de type Kelvin-Helmholtz entre les masses d'eaux méditerranéenne et atlantique. Ces instabilités marquent classiquement l'amorce de la cascade turbulente directe induisant localement un important mélange des masses d'eaux. Ces instabilités ne sont pas représentées dans la simulation NH-REF par manque de résolution effective. Elles ne pourraient être simulées explicitement dans une simulation hydrostatique de même résolution (45 m) dans cette région. Ces instabilités sont en effet associées à une vorticité relative d'axe horizontal qui ne peut être simulée sous une hypothèse hydrostatique. Un résultat important pour cette première simulation pouvant être qualifiée de \textit{LES} concerne les différences importantes entre le mélange simulé au moyen des maquettes NH-REF et NH-HR. L'intercomparaison montre à minima une très forte dépendance au protocole de simulation. La maquette NH-REF (de type \textit{RANS}\footnote{RANS: Reynolds Average Navier Stokes simulation, simulation basée sur une représentation implicite des "grandes structures turbulentes" à partir d'un schéma de fermeture turbulent.}) permet une simulation implicite du mélange et dépend pour cela entièrement du schéma de fermeture turbulente.\\
La maquette NH-HR montre donc l'importance d'une représentation explicite des "grandes structures turbulentes" dans des régions dynamiquement instables telles que la région du ressaut hydraulique. Elle donne de plus l'ordre de grandeur de la résolution horizontale (environ 45 m) et verticale (40 niveaux $\sigma$) permettant la simulation explicite des instabilités primaires.\\
Une première évaluation de la pertinence et de la qualité des structures instables simulées a pu être réalisée à partir d'observations publiées (Wesson and Gregg 1994). Un protocole d'observations dédié devra toutefois être plus spécifiquement mis en oeuvre pour espérer évaluer les caractéristiques principales de ces instabilités (localisation spatio-temporelle, amplitude...) et surtout pour quantifier les caractéristiques statiques du mélange qu'elles induisent. La campagne d'observations planifiée par le SHOM dans la région du détroit à la fin de l'été 2020 (soit un peu plus d'un an après la publication du présent rapport) devrait sans aucun doute répondre à un certain nombre de questions. Les maquettes numériques développées et évaluées dans le cadre du projet Gibraltar 16CR01 pourront être utilisées pour préparer cette campagne via par exemple la génération d'observations synthétiques.\\
Une discussion des principales perspectives associées à l'exploitation de la dynamique simulée d'une part et à l'évolution du protocole de simulation d'autre part est fournie au chapitre \ref{chapitreconclusion}.

%%%%%%%%%%%%%%%%%%%%%%%%%%%%%%%%%%%%%%%%%%%%%%%%%%%%%%%%%%%%%%%%%%%
%%%%%%%%%%%%%%%%%%%%%%%%%%%%%%%%%%%%%%%%%%%%%%%%%%%%%%%%%%%%%%%%%%%
%%% Généralisation du système de coordonnées
%%%%%%%%%%%%%%%%%%%%%%%%%%%%%%%%%%%%%%%%%%%%%%%%%%%%%%%%%%%%%%%%%%%
%%%%%%%%%%%%%%%%%%%%%%%%%%%%%%%%%%%%%%%%%%%%%%%%%%%%%%%%%%%%%%%%%%%
\newpage
\section{Généralisation du système de coordonnées verticales.\\ \textit{(Tâche 2)}}

Une pré-étude sur la pertinence d'une évolution du système de coordonnées verticales a été conduite par le professeur Eric Chassignet accueilli en 2017-2018 par le laboratoire d'Aérologie. Ce travail correspond à la Tâche 2 du contrat Gibraltar 16CR01 et a été décomposé en deux sous-tâches:
\begin{itemize}
\item{Analyse des solutions techniques \textit{(Tâche 2.1)}},
\item{Mise en œuvre et validation \textit{(Tâche 2.2)}}.
\end{itemize}
Le rapport final écrit par Eric Chassignet est fourni en Annexe.\\
Eric Chassignet a de surcroît co-organisé la première réunion des utilisateurs CROCO \textit{(CROCO USER-MEETING)} à Toulouse en Mai 2017.\\
 

%%%%%%%%%%%%%%%%%%%%%%%%%%%%%%%%%%%%%%%%%%%%%%%%%%%%%%%%%%%%%%%%%%%
%%%%%%%%%%%%%%%%%%%%%%%%%%%%%%%%%%%%%%%%%%%%%%%%%%%%%%%%%%%%%%%%%%%
%%% Chapitre Délivrables
%%%%%%%%%%%%%%%%%%%%%%%%%%%%%%%%%%%%%%%%%%%%%%%%%%%%%%%%%%%%%%%%%%%
%%%%%%%%%%%%%%%%%%%%%%%%%%%%%%%%%%%%%%%%%%%%%%%%%%%%%%%%%%%%%%%%%%%
\chapter{Délivrables, fourniture d'un \textit{produit numérique}.\\ \textit{(Tâche 3)}}

\label{chapitredelivrables}

\noindent\fbox{\noindent\begin{minipage}{1\textwidth}
Trois simulations ont été réalisées à partir de la maquette à haute résolution (45 m) NH-HR pour des marées de mortes-eaux, de vives-eaux et pour une marée d'amplitude moyenne. La dynamique des principales ondes de marée et la dynamique sub-tidale sont respectivement évaluées. 
\end{minipage}
}

\section{Protocole de simulation}
Début 2018, le contrat Gibraltar 16CR01 a fait l'objet d'un avenant portant sur la \textit{fourniture d'un produit numérique élaboré sur la base des maquettes numériques du détroit de Gibraltar telles que mises en place à la tâche 1}. Ce "produit numérique" est constitué de trois simulations numériques et leurs diagnostiques. Ces simulations numériques s'appuient sur la maquette numérique la plus réaliste et la mieux résolue présentée en section \ref{chapitremodele}.\ref{section3DNHNR}: la maquette NH-HR.
\\

\noindent Trois simulations \textit{(LES)} ont été réalisées avec la maquette NH-HR respectivement en condition de mortes-eaux (ME), vives-eaux (VE) et pour une marée intermédiaire (MM) entre ces deux marées extrêmes. 
Les dates de début et de fin de ces simulations sont données dans la table \ref{tab_dates_MIV}. 
Notons que ces périodes n'incluent pas les 6h de simulation hydrostatique nécessaires à l'initialisation (\textit{spin-up}). 
Notons également qu'il n'y a pas d'autres processus de forçage (eg. circulation grande échelle, surcotes...) que la marée imposée aux frontières du domaine.

\noindent La fréquence d'archivage des variables est de 1h sur l'ensemble du domaine, elle est moindre sur des zones d'intérêts pré-identifiées :
\begin{itemize}
\item Sur la région du Seuil de Camarinal (5°42'W - 5°48'W 35°52'N - 36°N) : sorties toutes les 2 minutes.
\item Sur l'entrée en Mer d'Alboran (5°3'W - 5°20'W 35°50'N - 36°10'N) : sorties toutes les 15 minutes.
\end{itemize}
Le forçage barotrope de marée est issu de simulations du modèle opérationnel de l'ENEA (voir section \ref{chapitremodele}.\ref{section3DNHNR}) et contient uniquement les 4 composantes de marée $M_2$, $S_2$, $K_1$ et $O_1$. 
La stratification est également issue de ce modèle aux dates d'initialisation de 
chacune des 3 situations.\\

\begin{table}[h]
        %\begin{minipage}{.6\textwidth}
        \centering
        \begin{tabular}{|c|c|}
                \hline
                Situation & Dates (UTC)\\
                \hline
                Mortes Eaux (VE) & 13/09/2017 16h00 - 15/09/17 17h00 \\
                %\hline
                Marées Moyenne (MM) & 16/09/2017 19h00 - 18/09/17 20h00  \\
                Vives Eaux (ME) & 19/09/2017 22h00 - 21/09/17 23h00  \\
                \hline
        \end{tabular}
        \captionof{table}{Périodes de simulation pour les 3 sitituations VE, MM et ME}
        \label{tab_dates_MIV}
        %\end{minipage}
\end{table}

\section{Analyse des composantes de la marée}
\label{mareedelivrable}
Les figures \ref{fig_maree_tar} et \ref{fig_maree_ceu} représentent les variations du niveau de la mer dans chaque
simulation par rapport aux données mesurées aux marégraphe de Tarifa et Ceuta. Ces données ont été mises à à disposition
gracieusement par l'institut public espagnol \textit{Puertos del Estado} et le programme \textit{GLOSS} (SHOM) \footnote{Nous remercions Puertos des Estados et le programme Gloss pour la fourniture de ces données marégraphiques. L'utilisation de ces résultats devra être également portée à leur connaissance.}.\\
Dans les 3 situations, les simulations sont en bon accord avec les observations marégraphiques. On observe toutefois de légers déphasages du cycle de marée (<1h), en avance ou au contraire en retard, selon les situations et les marégraphes. On observe aussi une différence d'amplitude entre les deux simulations et les données marégraphes résumées dans le tableau \ref{tab_rmse_MIV}. L'écart d'amplitude est de l'ordre de la dizaine de centimètres.

\begin{figure}[!h]
 	\includegraphics[width=\textwidth]{./sla_tarifa_all.png}
 	\caption{Anomalie du niveau de la mer (m) à Tarifa (36°N 5°36'W) enregistrée par le marégraphe 	(courbe noire), 
 	interpolée dans le champ de forçage (courbe rouge), dans CROCO (courbe bleu) dans la situation ME (a), MM (b) et VE (c). 
 	En tirets bleus le signal filtré à 25h dans CROCO.}
 	\label{fig_maree_tar}
\end{figure}

\begin{figure}[!h]
	\includegraphics[width=\textwidth]{./sla_ceuta_all.png}
 	\caption{Idem que figure \ref{fig_maree_tar} à Ceuta (35°54'N 5°19'W). 
 	Certaines mesures sont absentes pour la période VE.}
 	\label{fig_maree_ceu}
\end{figure}

Ces décalages ne sont pas identiques dans le cas du forçage par le MITgcm, 
soulignant la grande sensibilité de la modélisation numérique du détroit. On peut avancer comme raison à cela le fait que les bathymétries des 
2 modèles diffèrent. Les désaccords entre modèle(s) et marégraphes peuvent être dus également à un spectre de marée
incomplet voire à des processus manquants (effets de pression...). Des études complémentaires pourraient être envisagées
dans ces 2 cas. \\

\begin{table}[h]
        %\begin{minipage}{.6\textwidth}
        \centering
        \begin{tabular}{|c|c|c|c|c|}
                \hline
                Situation & \multicolumn{2}{c|}{Tarifa} &\multicolumn{2}{c|} {Ceuta}\\
                &CROCO &MitGCM&CROCO &MitGCM\\
                \hline
                Mortes Eaux (VE) &10 cm &7 cm &9 cm& 6 cm\\
                %\hline
                Marée Moyenne (MM) & 13 cm&10 cm &12 cm&9 cm \\
                Vives Eaux (ME) & 14 cm& 11 cm& 12 cm&9 cm \\
                \hline
        \end{tabular}
        \captionof{table}{Erreur quadratique moyenne de l'anomalie de niveau de la mer entre les simulations et les marégraphes.}
        \label{tab_rmse_MIV}
        %\end{minipage}
\end{table}

\section{Analyse de la dynamique aux échelles subtidales}
Le signal aux échelles subtidales est également ajouté à cette fourniture. Pour d'aussi courtes périodes de simulation (50 h), le signal de marée n'est pas filtrable par analyse harmonique 
(15 jours seraient nécessaires) ni par filtrage de Demerliac (filtre digital à 72h). Un filtrage par  moyenne glissante avec une fenêtre à 25 h a donc été implémenté et appliqué aux différentes variables (\textit{i.e.} niveau de la mer, courants, courants barotropes) issues de chaque simulation.\\
La qualité et la performance du filtrage ont été étudiées de manière statistique (écart-type temporel) en moyenne sur le domaine et aux marégraphes de Tarifa et Ceuta
où nous disposons de données sur la période étudiée. \\
En moyenne sur l'ensemble du domaine, en situation de mortes-eaux, 91\% de la variabilité du signal d'élévation du niveau de la mer est retiré par ce filtrage. 
En situation de marée moyenne (resp. de vives-eaux), ce filtrage permet de retirer 95\% (resp. 97\%) de la variabilité du signal d'élévation du niveau de la mer. \\
Aux marégraphes de Tarifa et Ceuta, les ordres de grandeur sont similaires et renseignés dans la table \ref{tab_stats_MIV}.
La variation temporelle du signal d'élévation résiduel du niveau de la mer est représentée dans les 3 situations sur les figures \ref{fig_maree_tar} et \ref{fig_maree_ceu}.

\begin{table}
        \centering
        \begin{tabular}{|c|c|c|c|}
                \hline
                Situation & \multicolumn{3}{c|}{Part de signal filtré}\\
                & Tout le domaine & Tarifa & Ceuta\\
                \hline
                Mortes Eaux (VE) &  91\% & 93\% & 86\%\\
                %\hline
                Marée Moyenne (MM) & 95\% & 90\% & 93\%  \\
                Vives Eaux (ME) & 96\% & 93\% & 95\% \\
                \hline
        \end{tabular}
        \captionof{table}{Performances du filtrage pour les 3 situations VE, MM et ME}
        \label{tab_stats_MIV}
        %\end{minipage}
\end{table}

\section{Consultation et récupération des délivrables (Dépôt)}

Comme stipulé dans le contrat Gibraltar 16CR01, l'ensemble des délivrables (\textit{produits numériques}) est mis à disposition sur la machine Datarmor accessible par les deux parties (\textit{SHOM} et \textit{Laboratoire d'Aérologie}). La table \ref{delivrables} résume l'ensemble des informations nécessaires à leur consultation et à leur récupération. \\
Le répertoire Livraison SHOM contient les fichiers des champs filtrés à 25 h (\textit{gibraltar\_nh-hr\_*.filt.nc}) pour les 3 situations (mortes eaux, vives eaux et marée moyenne) ainsi qu'un répertoire \textit{ Configuration\_NH-HR} où se trouvent les fichiers  \textit{cppdefs.h},  \textit{param.h} et \textit{croco.in.GIBRALTAR} servant à l'initialisation du modèle CROCO pour la maquette du détroit de Gibraltar 3D NH-HR. Les 3 simulations réalisées à partir de cette maquette sont également mises à disposition. Leur localisation est également décrite dans la table \ref{delivrables}. Les fichiers disponibles pour ces simulations sont :
\begin{itemize}
\item \textit{GBR\_NBQ\_his1h.nc} : Sorties toutes les heures sur tout le domaine des champs de courants horizontaux, courants barotropes et d'élévation de la surface libre 
\item \textit{GBR\_NBQ\_his3h.nc} : Sorties toutes les 3 heures sur tout le domaine des champs de masse volumique, salinité, température et vitesse verticale 
\item \textit{GBR\_NBQ\_his\_CS.nc} : Sorties toutes les 2 minutes dans la région du Seuil de Camarinal des champs de courants 3D, courants barotropes, élévation de la surface libre, masse volumique, salinité et température
\item \textit{GBR\_NBQ\_his\_XE.nc} : Sorties toutes les 15 minutes dans la région de la sortie est du détroit des champs de courants 3D, courants barotropes, élévation de la surface libre, masse volumique, salinité et température
\end{itemize}

\begin{table}
	\centering
	\begin{tabular}{|c|c|c|}
		\hline
		Contenu & Machine & Répertoire\\
		\hline
		Livraison SHOM & Datarmor & /home6/scratch/cnguyen/NHOMS/Livraison\_SHOM/ \\
		%\hline
 		Simulation mortes-eaux (ME)  & Datarmor & \$DIR/Run\_Gbr3d\_50mV2\_nbq\_ME2\_N40/ \\
		%\hline
 		Simulation vives-eaux (VE)  & Datarmor & \$DIR/Run\_Gbr3d\_50mV2\_nbq\_VE2\_N40/ \\
		%\hline
 		Simulation marée moyenne (MM)  & Datarmor & \$DIR/Run\_Gbr3d\_50mV2\_nbq\_IE2\_N40/ \\
 		\hline
	\end{tabular}
	\captionof{table}{Localisation des délivrables (\textit{produits numériques}).\\ Le répertoire d'origine est \$DIR = /home6/datawork/cnguyen/NHOMS/croco\_loc/ }
	\label{delivrables}
	%\end{minipage}
\end{table}

%%%%%%%%%%%%%%%%%%%%%%%%%%%%%%%%%%%%%%%%%%%
\chapter{Contrat Gibraltar 16CR01: principaux résultats et perspectives}
%%%%%%%%%%%%%%%%%%%%%%%%%%%%%%%%%%%%%%%%%%%
\label{chapitreconclusion}

Une hiérarchie de maquettes numériques de la région du détroit de Gibraltar a été implémentée dans le cadre du contrat Gibraltar 16CR01 avec pour principal objectif scientifique le développement d'une "simulation des grandes structures turbulentes``  \textit{(LES)} dans cette région. Une étude détaillée de la dynamique de \textit{fine échelle} a été proposée à partir de ces maquettes numériques. Plusieurs \textit{produits numériques} ont contractuellement été fournis et sont désormais disponibles sur le serveur CROCO du laboratoire d'Aérologie. Un rapport d'évaluation de la pertinence de coordonnées verticales généralisées a été fourni en Annexe.\\

\section{Principaux résultats}
 
La simulation des grandes structures turbulentes \textit{(LES)} de la région du détroit de Gibraltar ouvre des perspectives prometteuses pour l'étude de la cascade turbulente conduisant au mélange des masses d'eau. L'océan global est en effet connu pour n'abriter qu'un nombre vraisemblablement limité de régions présentant un mélange notable des masses d'eaux: régions peu-profondes localisées sur les plateaux continentaux ou dans les détroits, couches limites de fond et de surface abritant de violents cisaillements de courant, régions abritant sporadiquement des déferlements d'ondes de gravité ou des instabilités convectives... En dehors de ces "patchs" de mélange localisés dans l'espace et le temps, le mélange diapycnal dans la majeure partie de l'océan demeure faible (Munk et Wunsch, 1998). Cette double localisation des zones de mélange intense a plusieurs conséquences:
 \begin{itemize}
 	\item Le mélange diapycnal demeure difficile à quantifier et ses conséquences difficiles à évaluer: il est en effet intimement lié à des mécanismes de "fine échelle" mais il a des conséquences importantes sur les caractéristiques des masses d'eau et sur le maintien de la circulation à l'échelle globale.
 	\item Les observations directes du mélange et de la cascade turbulente directe dont il est une conséquence sont difficiles à réaliser, relativement récentes et restent très parcellaires.
 	\item La simulation directe du mélange diapycnal dans les modèles numériques est impossible et sa paramétrisation complexe.
 \end{itemize}
 La présente étude apporte ainsi plusieurs types de contributions pour progresser vers une meilleure connaissance du mélange diapycnal et de ses effets induits.\\
 
\noindent \textit{\textbf{Une région abritant un "patch" de mélange diapycnal intense}}\\
 Les résultats obtenus à partir de la maquette NH-HR montrent très clairement la présence d'une région de mélange intense légèrement à l'ouest du seuil de Camarinal, dans la région du ressaut hydraulique, et au dessus des principaux accidents bathymétriques. Même si l'amplitude des processus qui l'induisent est modulée par le cycle de marée, il n'en demeure pas moins que la région du détroit de Gibraltar est une zone privilégiée pour l'étude du mélange puisqu'elle abrite en permanence des "patches" de mélange intense entre deux masses constitutives de la circulation océanique générale. \\
 
\noindent\textit{\textbf{Une simulation originale des grandes échelles turbulentes (LES)}}\\
La simulation de la région du détroit avec des maquettes de mieux en mieux résolues a permis de montrer qu'une résolution de l'ordre de quelques dizaines de mètres (45 m pour 40 niveaux verticaux "$\sigma$") était à minima nécessaire pour modéliser explicitement les instabilités de type Kelvin-Helmholtz dans la région du détroit. Ces instabilités dites primaires initient localement la cascade turbulente directe conduisant au mélange des masses d'eau atlantique et méditerranéenne.\\
Le code CROCO et son coeur non-hydrostatique et non-Boussinesq ont ainsi été implémentés dans la région du détroit de Gibraltar (maquette NH-HR) afin de réaliser une simulation explicite des grandes structures turbulentes \textit{(LES)} dans la région du détroit. A notre connaissance, une telle simulation avec un code à surface libre explicite est une première.\\

\noindent\textit{\textbf{Une dynamique de "fine échelle"}}\\
La maquette NH-HR a de surcroît permis de confirmer les résultats obtenus par Bordois (2015) et Bordois et al. (2016, 2017) à partir de sections verticales 2D: caractéristiques spatiales et temporelles de l'onde solitaire, génération de solitons mode 1 et mode 2, génération d'ondes topographiques... Plusieurs réflexions des ondes solitaires sur les côtes bordant le détroit ont de plus été mises en évidence et comparées aux observations disponibles dans la région. La dynamique du ressaut hydraulique a pu être précisée et le développement d'instabilités primaires de type Kelvin-Helmholtz a été confirmée à l'interface entre les eaux méditerranéenne et atlantique. La génération de tourbillons d'extension réduite au passage des solitons internes a été confirmée et le rôle déstructurant de ces tourbillons sur les trains de solitons est actuellement en cours d'étude.
La qualité de la simulation des ondes et courants de marée dans cette région d'extension réduite et très active dynamiquement a été évaluée et des stratégies de simulation ont été proposées et implémentées pour améliorer la précision numérique des prévisions.\\

\section{Perspectives}

 \noindent\textit{\textbf{Perspectives numériques}}\\
 Comme discuté à plusieurs reprises dans le présent rapport, les différentes maquettes implémentées et exploitées dans le cadre du projet Gibraltar 16CR01 ne sont pas figées mais en constante évolution. Les évolutions en cours sont multiples:
 \begin{itemize}
     \item Le code communautaire CROCO et son coeur numérique non-hydrostatique et non-Boussinesq CROCO-NBQ sont aménés à évoluer. De nouveaux schémas d'advection, de fermeture turbulente LES ou plus fondamentalement de couplage des modes numériques (associés au \textit{mode splitting}) sont en cours de développement et de test.
     \item La quête pour une plus grande efficacité du code a déjà permis des gains importants depuis la première implantation de l'algorithme non-hydrostatique, non-Boussinesq dans le code de recherche SNBQ. Elle se poursuit actuellement avec le portable du code sur des machines à structure dite hétérogène (CPU-GPU).
     \item En lien avec les futurs gains en efficacité de calcul, la maquette à haute résolution NH-HR est aussi amenée à évoluer vers plus de réalisme: les conséquences du forçage par la circulation générale et la marée étant correctement appréhendées, le forçage atmosphérique peut désormais être introduit. Le domaine simulé est de plus en cours d'extension, en particulier vers l'ouest.
 \end{itemize}
 
\noindent\textit{\textbf{Portabilité de l'approche "LES"}}\\
Le coeur numérique non-hydrostatique, non-Boussinesq est implanté dans le code communautaire CROCO héritant ainsi de toutes les fonctionnalités (en termes de pré et de post-traitements) et de l'importante communauté d'utilisateurs de ROMS-AGRIF. L'étude réalisée dans le cadre du projet Gibraltar 16CR01 montre l'adéquation originale de ce code avec le basculement de la modélisation numérique de l'océan vers la simulation des grandes structures turbulentes (\textit{LES}). Cette approche numérique peut aujourd'hui être aisément utilisée dans d'autres régions du globe pour lesquelles une "simulation des grandes structures turbulences" pourrait permettre d'ouvrir d'intéressantes perspectives en termes de dynamique de l'océan global:
\begin{itemize}
	\item La région littorale dans laquelle vagues, niveau de la mer et courants sont intimement liés via des mécanismes de fine (voire de très fine) échelle.
	\item Les régions de détroit, de seuils et autres fjords sont potentiellement associées, comme le montre la présente étude, à des processus de contrôle hydraulique, de générations d'ondes de gravité et finalement de mélange intense.
	\item La couche d'Ekman présente de forts cisaillements de courants et donc potentiellement des instabilités de fine échelle.
	\item Les régions de "formations d'eaux denses" (zones de convection profonde, rebords des talus continentaux...) présentent des processus induisant potentiellement de fortes accélérations verticales ainsi qu'un important mélange induit.
	\item Les marées internes sont associées dans leurs zones de génération et de déferlement à d'intenses \textit{patchs} de mélange...
\end{itemize}

\noindent\textit{\textbf{Raffinement possible des modèles de circulation générale (OGCM)}}\\
Le code communautaire CROCO a en outre hérité des fonctionnalités d'imbrication et de zoom interactifs\footnote{En anglais two-way nesting.} de ROMS-AGRIF. La maquette NH-HR ouvre par conséquent de nombreuses perspectives de raffinement des OGCM dans des régions connues pour abriter sporadiquement ou continuellement des processus de fine échelle ayant un impact directe sur la circulation et les masses d'eau. Les modèles numériques concernés ne résolvant pas la physique de ces fines échelles (non-hydrostatique en particulier...) et ne disposant pas non plus de la résolution nécessaire (quelques dizaines de mètres seulement), l'imbrication de zooms est la seule alternative à la paramétrisation brutale de l'ensemble des processus non résolus.\\
Un exemple d'application directe de la présente étude concerne l'imbrication de la maquette NH-HR dans un modèle global de la circulation méditerranéenne.\\
 
\noindent\textit{\textbf{Le mélange diapycnal et ses conséquences}}\\
Le raffinement local de la circulation générale ouvre donc aussi des perspectives pour une meilleure représentation du "mélange" dans les modèles numériques de bassin. On peut en effet supposer que le mélange induit dans des régions telles que la région du ressaut hydraulique à Gibraltar sera plus facilement localisé et ainsi beaucoup mieux quantifié si l'on est capable de simuler explicitement les plus grandes structures turbulentes dans cette région. Leur modélisation implicite dans les simulations RANS réalisées jusqu'à présent repose en effet entièrement sur des schémas de fermeture turbulente représentant les conséquences des instabilités primaires. La simulation explicite d'une partie au moins de ces instabilités (les "structures turbulentes de grande échelle") permet une comparaison avec des observations directes ainsi qu'une analyse plus fine de leurs conséquences. \\
Dans la région du détroit de Gibraltar toujours, nous disposons enfin avec la maquette NH-HR d'un outil d'étude de la formation du jet méditerranéen. Les mécanismes de génération du contenu en vorticité potentielle peuvent ainsi être étudiés à partir d'une maquette d'extension horizontale légèrement plus étendue. Une telle étude devrait permettre de mieux comprendre comment les différents mécanismes induisant un mélange notable peuvent modifier le contenu en vorticité potentielle de ce courant, bien en amont de la génération des instabilités baroclines auxquelles il est finalement soumis.\\


\noindent\textit{\textbf{La complémentarité avec les futures campagnes d'observations}}\\
Les simulations numériques et l'étude de la dynamique de \testit{fine échelle} ont montré la nécessité d'une simulation des \textit{grandes structures turbulentes (LES)} dans la région du détroit de Gibraltar et dans la foulée l'adéquation du protocole original de modélisation basé sur le code CROCO. Les processus de \textit{fine échelle} (ressauts hydrauliques, trains d'ondes solitaires, mécanisme de génération du jet méditerranéen, ondes de sillage, structures tourbillonnaires, instabilités primaires de Kelvin-Helmholtz...) ont pu être simulés explicitement. A notre connaissance, cette dynamique de \textit{fine échelle} n'a pas à ce jour fait l'objet d'observations in-situ dédiées dans la région du détroit. Une campagne d'observations permettrait d'évaluer précisément la pertinence de ce protocole de modélisation et des choix numériques. Elle ouvrirait aussi la porte à de nouveaux développements en mettant en évidence les limites du système actuel et permettrait finalement de préciser la localisation, la périodicité et l'amplitude des processus et autres mécanismes de \textit{fine échelle} mis en évidence.

\newpage
 
%%%%%%%%
%%% Bibliographie
%%%%%%%
%\newpage
\addcontentsline{toc}{chapter}{Bibliographie}
\nocite{Baines1995}
\nocite{Bordois15}
\nocite{Bordois16}
\nocite{Bordois17}
\nocite{BS84}
\nocite{FA1988}
\nocite{FA1986}
\nocite{Garett90}
\nocite{Vazquez2006}
\nocite{SG2011}
%\nocite{Sannino2014}
\nocite{Naranjo2014}
\nocite{Brandt1996}
\nocite{Sannino2002}
\nocite{Sannino2004}
\nocite{Sannino2007}
\nocite{Sannino2009}
\nocite{Sannino2009b}
\nocite{Sannino2015}
\nocite{Izquierdo2001}
\nocite{Auclair2018}
\nocite{Auclair2011}
\nocite{SN2015}
\nocite{CW90}
\nocite{Beth79}
\nocite{Bray95}
\nocite{FA1988}
\nocite{Bormans1989}
\nocite{Vlasenko2009}
\nocite{Bryden94}
\nocite{SG2008}
\nocite{Wesson94}
\nocite{Dossmann2012}
\nocite{Grinstein2007}
\nocite{Bruno2002}
\nocite{tpxo8}
\nocite{alpers85}
\nocite{Debreu2012}
\nocite{Penven2006}
\nocite{Shchepetkin2003}
\nocite{Shchepetkin1998}
\nocite{S2005}
\nocite{Marchesiello2001}
\nocite{Marchesiello2003}
\nocite{Munk1998}

%\begin{multicols}{2}
%\bibliographystyle{plain}
\bibliographystyle{acm}
\bibliography{bibliography}
%\end{multicols}

\newpage
	 \null
\newpage

 \chapter*{Annexe: rapport rédigé par E. Chassignet \textit{(Tâche 2)}}
\addcontentsline{toc}{chapter}{Annexe: rapport rédigé par E. Chassignet \textit{(Tâche 2)}}
\newpage
	 \null
\newpage
\includepdf[pages=-]{RapportSHOMEChassignet2017.pdf}

\end{document}







